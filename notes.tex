\documentclass[a4paper,11pt]{article}

\usepackage[utf8]{inputenc}
\usepackage[T1]{fontenc}
\usepackage[english]{babel}
\usepackage{graphicx}
\usepackage{amsmath,amssymb,amsthm,amsopn}
\usepackage{mathrsfs}
\usepackage{graphicx}
\usepackage{array}
\usepackage{makecell}


\usepackage{hyperref}
\hypersetup{
    colorlinks=true,
    linkcolor=blue,
    citecolor=red,
}

%\usepackage[top=1cm,bottom=1cm]{geometry}
%\usepackage{listings}
%\usepackage{xcolor}

\usepackage{tikz}

% Tikz style

\tikzset{round/.style={circle, draw=black, very thick, scale = 0.7}}
\tikzset{arrow/.style={->, >=latex}}
\tikzset{dashed-arrow/.style={->, >=latex, dashed}}

\newtheoremstyle{break}%
{}{}%
{\itshape}{}%
{\bfseries}{}%  % Note that final punctuation is omitted.
{\newline}{}

\newtheoremstyle{sc}%
{}{}%
{}{}%
{\scshape}{}%  % Note that final punctuation is omitted.
{\newline}{}

\theoremstyle{break}
\newtheorem{thm}{Theorem}[section]
\newtheorem{lm}[thm]{Lemma}
\newtheorem{prop}[thm]{Proposition}
\newtheorem{cor}[thm]{Corollary}

\theoremstyle{sc}
\newtheorem{exo}{Exercice}

\theoremstyle{definition}
\newtheorem{defi}[thm]{Definition}
\newtheorem{ex}[thm]{Example}

\theoremstyle{remark}
\newtheorem{rem}[thm]{Remark}

% Math Operators

\DeclareMathOperator{\Card}{Card}
\DeclareMathOperator{\Gal}{Gal}
\DeclareMathOperator{\Id}{Id}
\DeclareMathOperator{\Img}{Im}
\DeclareMathOperator{\Ker}{Ker}
\DeclareMathOperator{\Minpoly}{Minpoly}
\DeclareMathOperator{\Mod}{mod}
\DeclareMathOperator{\Ord}{Ord}
\DeclareMathOperator{\ppcm}{ppcm}
\DeclareMathOperator{\Tr}{Tr}
\DeclareMathOperator{\Vect}{Vect}

% Shortcuts

\newcommand{\dE}{\partial(E)}
\newcommand{\dF}{\partial(F)}
\newcommand{\dG}{\partial(G)}
\newcommand{\diff}{\mathop{}\!\mathrm{d}}
\newcommand{\eg}{\emph{e.g. }}
\newcommand{\emb}{\hookrightarrow}
\newcommand{\embed}[2]{\phi_{#1\hookrightarrow#2}}
\newcommand{\ent}[2]{[\![#1,#2]\!]}
\newcommand{\ie}{\emph{i.e. }}
\newcommand{\ps}[2]{\left\langle#1,#2\right\rangle}





% opening
\title{Some notes on the compatibility of embeddings using Allombert's algorithm}
\author{}



\begin{document}

\maketitle

%\begin{abstract}

%\end{abstract}

%\tableofcontents

%\clearpage

\section{Towers of $\ell$-adic extensions, $\ell\neq2$}
\label{sec:ladic}

Let $p$ be a prime number, we are interested in the tower of $\ell$-adic extensions of
$k=\mathbb{F}_p$, where $\ell\notin\left\{ 2, p
\right\}$ is an other prime
number. For each $i$, we denote by
$\mathbb{F}_{p^{\ell^i}}$ the extension of degree $\ell^i$, $\zeta_{\ell^i}$ a
$\ell^i$-th primitive root of unity and $r_{\ell^i}$ the integer such that
$k(\zeta_{\ell^i})\cong\mathbb{F}_{p^{r_{\ell^i}}}$. We also denote by
$A_{\ell^i}=\mathbb{F}_{p^{\ell^i}}\otimes
k(\zeta_{\ell^i})$ the algebra used to solve Hilbert 90 problems, and
$\alpha_{\ell^i}\in A_{\ell^i}$ the
solution that we use.

We have $r_{\ell}=\omega_{(\mathbb{Z}/\ell\mathbb{Z})^\times}(p)$, where
$\omega_{(\mathbb{Z}/\ell\mathbb{Z})^\times}(p)$ denotes the order of $p$ in
$(\mathbb{Z}/\ell\mathbb{Z})^\times$. If we denote by $\upsilon=v_\ell(p^{r_\ell}-1)$
the $\ell$-adic valuation of $p^{r_\ell}-1$, we get 
\[
  r_\ell = r_{\ell^2} = \cdots = r_{\ell^{\upsilon}},
\]
and if $i\geq\upsilon$, we have $r_{\ell^{i+1}}=\ell r_{\ell^i}$.

Assume we have an integer $n\in\mathbb{N}$, a root
$\zeta_{\ell^n}$ and a solution of Hilbert 90 $\alpha_{\ell^n}\in A_{\ell^n}$. We are interested in
constructing the embedding $\mathbb{F}_{p^{\ell^n}}\emb
\mathbb{F}_{p^{\ell^{n+1}}}$, using Allombert's algorithm. We
first compute 
\[
  g = \gcd(\pi_{\ell^n}(x^\ell), \Phi_{\ell^{n+1}}(x)),
\]
where $\pi_{\ell^n}$ is the minimal polynomial of $\zeta_{\ell^n}$ over $k$ and
$\Phi_{\ell^{n+1}}$ is the $\ell^{n+1}$-th cyclotomic polynomial. We set
$\pi_{\ell^{n+1}}$ to be an irreducicle factor of $g$ and we take
$\zeta_{\ell^{n+1}}$ a root
of $\pi_{\ell^{n+1}}$. We then define
$A_{\ell^{n+1}}=\mathbb{F}_{p^{\ell^{n+1}}}\otimes k(\zeta_{\ell^{n+1}})$ and
\[
\begin{array}{cccc}
  N_{\ell^{n+1}/\ell^n,\, \bm{\ell^{n+1}}}: & A_{\bm{\ell^{n+1}}} & \to &
  A_{\bm{\ell^{n+1}}} \\
  & \gamma & \mapsto & \prod_{j=0}^{\ell-1} (1 \otimes
  \sigma^{jr_{\ell^n}})(\gamma)
\end{array}
\]
where $\sigma$ is the Frobenius automorphism of $k$. If the domain of the map is
clear within the given context, we will omit the second index and only note
$N_{\ell^{n+1}/\ell^n}$. We note that, in particular, given
$\gamma=x\otimes y\in A_{\ell^{n+1}}$ a simple tensor, if
$r_{\ell^{n+1}}=r_{\ell^n}$, we have $N_{\ell^{n+1}/\ell^n}(\gamma) =
\gamma^\ell$, and if $r_{n+1}=\ell r_n$, we have $N_{\ell^{n+1}/\ell^n}(\gamma) = x^\ell\otimes
N_{k(\zeta_{\ell^{n+1}})/k(\zeta_{\ell^n})}(y)$, where
$N_{k(\zeta_{\ell^{n+1}})/k(\zeta_{\ell^n})}$ is the
norm of the extension $k(\zeta_{\ell^{n+1}})$ over $k(\zeta_{\ell^n})$. More generally, if $\gamma\in
A_{\ell^{n+1}}$ is not a simple tensor, we have that
\[
  (1\otimes\sigma^{r_{\ell^n}})(N_{\ell^{n+1}/\ell^n}(\gamma)) =
  N_{\ell^{n+1}/\ell^n}(\gamma),
\]
so we still have 
\[
  N_{\ell^{n+1}/\ell^n}(\gamma)\in \mathbb{F}_{p^{\ell^{n+1}}}\otimes \phi_{k(\zeta_{\ell^n})\emb
k(\zeta_{\ell^{n+1}})}(k(\zeta_n)).
\]

We indeed embed $k(\zeta_{\ell^n})$ in $k(\zeta_{\ell^{n+1}})$ by sending
$\zeta_{\ell^n}$ to
$\zeta_{\ell^{n+1}}^\ell$ and we denote by $\phi_{k(\zeta_{\ell^n})\emb
k(\zeta_{\ell^{n+1}})}$ this embedding. We also note
\[
  \Phi_{A_{\ell^n}\emb
  A_{\ell^{n+1}}}=\phi_{\mathbb{F}_{p^{\ell^n}}\emb\mathbb{F}_{p^{\ell^{n+1}}}}\otimes\phi_{k(\zeta_{\ell^n})\emb
k(\zeta_{\ell^{n+1}})}
\]
an embedding from the algebra $A_{\ell^n}$ in $A_{\ell^{n+1}}$, where $\phi_{\mathbb{F}_{p^{\ell^n}}\emb\mathbb{F}_{p^{\ell^{n+1}}}}$ is some
embedding of $\mathbb{F}_{p^{\ell^n}}$ in $\mathbb{F}_{p^{\ell^{n+1}}}$.
With that choice we have that
\[
  N_{\ell^{n+1}/\ell^n}(1\otimes\zeta_{\ell^{n+1}})=\Phi_{A_{\ell^n}\emb
  A_{\ell^{n+1}}}(1\otimes\zeta_{\ell^n}).\footnote{If $r_{\ell^{n+1}}=r_{\ell^n}$,
  $N_{\ell^{n+1}/\ell^n}(1\otimes\zeta_{n+1})=1\otimes\zeta_{n+1}^\ell$. If
  $r_{\ell^{n+1}}=\ell r_{\ell^n}$, it means that
  there are no roots of $X^\ell-\zeta_{\ell^n}$ in $k(\zeta_{\ell^n})$. Since there are $\ell$-roots of unity in
  $k(\zeta_{\ell^n})$, the polynomial $X^\ell-\zeta_{\ell^n}$ is either irreducible or splits
  into linear polynomial over $k(\zeta_{\ell^n})[X]$. The polynomial $X^\ell
  -\zeta_{\ell^{n+1}}^\ell=X^\ell-\phi_{\mathbb{F}_{p^{\ell^n}}\emb
  \mathbb{F}_{p^{\ell^{n+1}}}}(
  \zeta_{\ell^n})$ is
  the minimal polynomial of $\zeta_{\ell^{n+1}}$ over $\phi_{\mathbb{F}_{p^{\ell^n}}\emb
  \mathbb{F}_{p^{\ell^{n+1}}}}(k(\zeta_{\ell^n}))$ and the product of its
  roots (\ie the norm of $\zeta_{\ell^{n+1}}$) is (up to the sign) the constant
  coefficient, so
  $N_{\ell^{n+1}/\ell^n}(1\otimes\zeta_{\ell^{n+1}})=\Phi_{A_{\ell^n}\emb
  A_{\ell^{n+1}}}(1\otimes\zeta_{\ell^n})$.}
\]
We also see that $N_{\ell^{n+1}/\ell^n}$ and $\sigma\otimes1$ commute, and for any $\alpha,
\beta\in A_{\ell^{n+1}}$, we have that \[
  N_{\ell^{n+1}/\ell^n}(\alpha\beta) =
  N_{\ell^{n+1}/\ell^n}(\alpha)N_{\ell^{n+1}/\ell^n}(\beta).
\]
So if $\beta\in A_{\ell^{n+1}}$ is a nonzero solution of the Hilbert 90 problem
in $A_{\ell^{n+1}}$ for
$1\otimes\zeta_{\ell^{n+1}}$, \ie
\[
  (\sigma\otimes 1)(\beta) = (1\otimes\zeta_{\ell^{n+1}})\beta,
\]
then $N_{\ell^{n+1}/\ell^n}(\beta)$ is a nonzero solution of Hilbert 90 for
$\Phi_{A_{\ell^n}\emb A_{\ell^{n+1}}}(1\otimes\zeta_{\ell^n})$:
\begin{align*}
  (\sigma\otimes1)(N_{\ell^{n+1}/\ell^n}(\beta)) &= N_{\ell^{n+1}/\ell^n}((\sigma\otimes1)(\beta)) \\
  &= N_{\ell^{n+1}/\ell^n}(1\otimes\zeta_{\ell^{n+1}})N_{\ell^{n+1/\ell^n}}(\beta) \\
  &= \Phi_{A_{\ell^n}\emb A_{\ell^{n+1}}}(1\otimes\zeta_n)N_{\ell^{n+1}/\ell^n}(\beta).
\end{align*}
This also implies that $N_{\ell^{n+1}/\ell^n}(\beta)$ is in fact in
$\Phi_{A_{\ell^n}\emb A_{\ell^{n+1}}}(A_{\ell^n})$, so we have that
\[
  N_{\ell^{n+1}/\ell^n}(\beta) = (1\otimes a)\Phi_{A_{\ell^n}\emb
  A_{\ell^{n+1}}}(\alpha_n),
\]
with $1\otimes a\in \Phi_{A_{\ell^n}\emb A_{\ell^{n+1}}}(1\otimes
k(\zeta_{\ell^n}))$ and $a\neq0$, because the solutions of Hilbert 90 in
$\Phi_{A_{\ell^n}\emb A_{\ell^{n+1}}}(A_{\ell^n})$
form a $\phi_{k(\zeta_{\ell^n})\emb k(\zeta_{\ell^{n+1}})}(k(\zeta_{\ell^n}))$ vector space of dimension $1$. We now want to find an
element $1\otimes b\in1\otimes k(\zeta_{\ell^{n+1}})$ such that
\[
N_{\ell^{n+1}/\ell^n}(1\otimes b)=1\otimes a^{-1}.
\]
We distinguish two
cases: $r_{n+1}=\ell r_n$ and $r_{n+1}=r_n$.
Let us first assume that $r_{\ell^{n+1}}=\ell r_{\ell^n}$, it means that
\[
  \phi_{k(\zeta_{\ell^n})\emb k(\zeta_{\ell^{n+1}})}(k(\zeta_{\ell^n}))\subsetneq
k(\zeta_{\ell^{n+1}})
\]
and that $N_{\ell^{n+1}/\ell^{n}}$ operates as the norm
$N_{k(\zeta_{n+1})/k(\zeta_n)}$ on the right side of $A_{\ell^{n+1}}$. Since this norm
is surjective we can find the  element $b$ such that $N_{\ell^{n+1}/\ell^n}(1\otimes b)=1\otimes a^{-1}$ and
we set 
\[
  \alpha_{\ell^{n+1}}\eqdef (1\otimes b)\beta,
\]
in order to have
\[
  N_{\ell^{n+1}/\ell^n}(\alpha_{\ell^{n+1}})=\Phi_{A_{\ell}\emb
  A_{\ell^{n+1}}}(\alpha_{\ell^n}).
\]
In order to compute $b$ we remark that either the element
$a^{-1}$ is a $\ell$-th power in $\phi_{k(\zeta_{\ell^m})\emb
k(\zeta_{\ell^{n+1}})}(k(\zeta_{\ell^n}))$, or the polynomial
$X^\ell-a^{-1}$ is irreducible in $\phi_{k(\zeta_{\ell^m})\emb
k(\zeta_{\ell^{n+1}})}(k(\zeta_{\ell^n}))[X]$, so computing an $\ell$-th
root of $a^{-1}$ is sufficient. In the case where $r_{\ell^{n+1}}=r_{\ell^n}$,
there is only one of the $\ell$ possible roots $\zeta_{\ell^{n+1}}$ such that
$a$ is a $\ell$-th power, at least for $\ell\neq2$.

More generally, in the case of embeddings 
$\mathbb{F}_{p^{\ell^m}}\emb\mathbb{F}_{p^{\ell^n}}$, with $m<n$ two integers, we can similarly define 
\[
\begin{array}{cccc}
  N_{\ell^{n}/\ell^m,\, \bm{\ell^{n}}}: & A_{\bm{\ell^{n}}} & \to &
  A_{\bm{\ell^{n}}} \\
  & \gamma & \mapsto & \prod_{j=0}^{\ell^{n-m}-1} (1 \otimes
  \sigma^{jr_{\ell^m}})(\gamma)
\end{array}
\]
and
\[
  \Phi_{A_{\ell^m}\emb A_{\ell^n}}(\alpha_{\ell^m})\eqdef
  N_{\ell^n/\ell^m}(\alpha_{\ell^{n}}).
\]
We see that the norms are transitive, \ie for each triple of integers $m, n, o$
such that $m<n<o$,
we have
\[
  N_{\ell^o/\ell^m}=N_{\ell^{n}/\ell^{m}}\circ N_{\ell^o/\ell^n}.
\]
These ``norms'' commute with the morphisms of algebra $\Phi$, so we obtain 
\[
  \Phi_{A_{\ell^m}\emb A_{\ell^o}}(\alpha_{\ell^m}) = \Phi_{A_{\ell^{n}}\emb
A_{\ell^{o}}}\circ \Phi_{A_{\ell^{m}}\emb
A_{\ell^{n}}}(\alpha_{\ell^m}).
\]
And as a corollary we obtain that the embeddings between the finite fields of
the $\ell$-adic tower obtained via $\Phi$ are compatible. 

\section{Composita using norms, $\ell\neq2$}

The considerations in Section~\ref{sec:ladic} generalize to the case of
embeddings between composita, but depending on the nature of the composita,
there can be more or less additionnal work.

\subsection{Composita without new primes}
\label{sec:comp-norm-wo}

Let $m, n\in\mathbb{N}$
two integers such that $m\,|\,n$ and such that for all prime $\ell$ such that
$\ell\,|\,n$, we also have $\ell\,|\,m$. In other words, we want $m$ and $n$ to
have the same set of primes in their factorization. Our goal is to extend the
results of Section~\ref{sec:ladic} to the embedding
$\mathbb{F}_{p^m}\emb\mathbb{F}_{p^n}$. Let us note 
\[
  m = \prod_i \ell_i^{\alpha_i}
\]
the factorization of $m$ into primes. For now, we also assume that for all $i$,
we have 
\[
  \alpha_i\geq h(\ell_i)
\]
where, for any prime $\ell$, 
\[
  h(\ell) = v_\ell(p^{r_\ell}-1)
\]
with $v_\ell$ being the $\ell$-adic valuation and
$r_\ell=\omega_{(\mathbb{Z}/\ell\mathbb{Z})^\times}(p)$ is the multiplicative
order of $p$ in the group $(\mathbb{Z}/\ell\mathbb{Z})^\times$, meaning that
$r_\ell$ is the number such that we have 
\[
  k(\zeta_\ell)=\mathbb{F}_p(\zeta_\ell)=\mathbb{F}_{p^{r_\ell}},
\]
where $\zeta_\ell$ refers to a $\ell$-th primitive root of unity.
In the end, these numbers mean that given an integer $\nu\in\mathbb{N}$, if
$1\leq\nu\leq h(\ell)$, we have
\[
  k(\zeta_{\ell^\nu}) = \mathbb{F}_{p^{r_\ell}},
\]
and if $\nu\geq h(\ell)$, we have
\[
  k(\zeta_{\ell^\nu}) = \mathbb{F}_{p^{\ell^{\nu-h(\ell)}\times r_\ell}}.
\]
Since $r_\ell\leq\ell-1$, these conditions on $n$ and $m$ are equivalent to the fact that
\[
  [k(\zeta_n):k(\zeta_m)] = n/m.
\]

We define $A_n=\mathbb{F}_{p^n}\otimes k(\zeta_n)$ and
\[
\begin{array}{cccc}
  N_{n/m}: & A_n & \to & A_n \\
  & \gamma & \mapsto & \prod_{j=0}^{n/m-1} (1\otimes \sigma^{jd})(\gamma)
\end{array}
\]

where $d$ is the degree of the extension $k(\zeta_m)$ over $k$. We embed
$k(\zeta_m)$ in $k(\zeta_n)$ by sending $\zeta_m$ to $\zeta_n^{n/m}$. With that
choice we have that
\[
  N_{n/m}(1\otimes\zeta_n) = \Phi_{A_m\emb A_n}(1\otimes\zeta_m),
\]
with 
\[
  \Phi_{A_m\emb
  A_n}=\phi_{\mathbb{F}_{p^m}\emb\mathbb{F}_{p^n}}\otimes\phi_{k(\zeta_m)\emb
  k(\zeta_n)}.
\]
Indeed, if $n = \ell m$ for a prime $\ell$ dividing $m$, the arguments of
Section~\ref{sec:ladic} still hold and the more general case follows by
transitivity of the ``norm'' $N$, \ie given $m\,|\,n\,|\,o$, we have
$N_{o/m}=N_{n/m}\circ N_{o/n}$. As in the previous section, we see that
$N_{n/m}$ is multiplicative and that if $\beta\in A_n$ is a nonzero solution of
the Hilbert 90 problem in $A_n$ for $1\otimes\zeta_n$, then $N_{n/m}(\beta)$ is
an element of $A_m$ and is a nonzero solution of Hilbert 90 for
$1\otimes\zeta_m$. Therefore, as the solutions of Hilbert 90 in $A_m$ for
$1\otimes\zeta_m$ form a dimension one $k(\zeta_m)$-vector space and since
$N_{n/m}$ is the usual norm of the extension $k(\zeta_n)$ over $k(\zeta_m)$
(which is surjective) on the right part of $A_n=\mathbb{F}_{p^n}\otimes
k(\zeta_n)$, we can conclude that the compatibility can be achieved, as in
Section~\ref{sec:ladic}\footnote{So here we just say that we can achieve to
  obtain an element $\alpha_n\in A_n$ such that
  $N_{n/m}(\alpha_n)=\Phi(\alpha_m)$ for a certain $m\,|\,n$, and that it is
  possible for all choice of $m$, but it \emph{does not} mean that we can do it
simultaneously for all $m$.}.

Now, if we have one $i$ with $\alpha_i\leq h(\ell_i)$, it is not clear how to
extend the results of the last section.

\subsection{Composita with new primes}
\label{sec:comp-norm-w}


\section{Composita using chinese remainder}
\label{sec:comp-crt}

As we saw in Section~\ref{sec:comp-norm-wo}, if we use only norms, we can 
compatibly embed finite fields $\mathbb{F}_{p^m}\emb\mathbb{F}_{p^n}$ if
$[k(\zeta_n):k(\zeta_m)]=n/m$. We will now present a different technique that
still uses norms on the different towers of $\ell$-adic extensions, but
constructs the composita using chinese remainder. Let us start with a simple
example. We let $\ell\in\mathbb{N}$ and $m\in\mathbb{N}$ be two different prime numbers and we will
investigate the compatibility of the embeddings between $\mathbb{F}_{p^\ell}$,
$\mathbb{F}_{p^{\ell m}}$ and $\mathbb{F}_{p^{\ell^2 m^2}}$. Let $u_1, u_2, v_1,
v_2\in \mathbb{N}$ be integers such that
\[
  u_1\ell+u_2m=1
\]
\[
  v_1\ell^2+v_2m^2=1.
\]
Let $\zeta_\ell$ (resp. $\zeta_m$) be a $\ell$-th (resp. $m$-th) primitive root
of unity. Using the construction described in Section~\ref{sec:ladic}, we denote
by $\zeta_{\ell^2}$ (resp. $\zeta_{m^2}$) an $\ell^2$-th (resp. $m^2$-th)
primitive root of unity and $\alpha_{\ell^2}\in A_{\ell^2}$ (resp.
$\alpha_{m^2}\in A_{m^2}$) a nonzero solution of Hilbert $90$ in $A_{\ell^2}$
for $1\otimes\zeta_{\ell^2}$ (resp. in $A_{m^2}$ for $1\otimes\zeta_{m^2}$) such that
\[
  N_{\ell^2/\ell}(\alpha_{\ell^2})=\Phi_{A_\ell\emb A_{\ell^2}}(\alpha_\ell)
\]
\[
  N_{m^2/m}(\alpha_{m^2})=\Phi_{A_m\emb A_{m^2}}(\alpha_m).
\]
We also define\footnote{This is an abuse of notation.}
\[
  \zeta_{\ell m}=\zeta_\ell^{u_2}\zeta_{m}^{u_1}
\]
and
\[
  \zeta_{\ell^2 m^2}=\zeta_{\ell^2}^{v_2}\zeta_{m^2}^{v_1},
\]
so that $\zeta_{\ell m}^\ell = \zeta_m$, $\zeta_{\ell m}^\ell=\zeta_m$,
$\zeta_{\ell^2m^2}^{\ell^2}=\zeta_{m^2}$ and $\zeta_{\ell^2
m^2}^{m^2}=\zeta_{\ell^2}$, and
\[
  \alpha_{\ell m}=\Phi_{A_\ell\emb A_{\ell m}}(\alpha_\ell)^{u_2}\Phi_{A_m\emb
    A_{\ell m}}(\alpha_m)^{u_1}
\]
\[
  \alpha_{\ell^2 m^2}=\Phi_{A_{\ell^2}\emb A_{\ell^2
  m^2}}(\alpha_{\ell^2})^{v_2}\Phi_{A_{m^2}\emb
  A_{\ell^2 m^2}}(\alpha_{m^2})^{v_1}
\]
where the images of the maps $\Phi$ are obtained using Allombert's algorithm.

With these definitions, we get:
\begin{align*}
  \alpha_{\ell m}^m &= \Phi_{A_\ell\emb A_{\ell m}}(\alpha_\ell)^{u_2m}\Phi_{A_m\emb
    A_{\ell m}}(\alpha_m)^{u_1m} \\
  &= \Phi_{A_\ell\emb A_{\ell m}}(\alpha_\ell)^{1-u_1\ell}\Phi_{A_m\emb
    A_{\ell m}}(\alpha_m)^{u_1m} \\
    &= \left( \cfrac{\Phi_{A_m\emb A_{\ell m}}(\alpha_{m}^m)}{\Phi_{A_\ell\emb A_{\ell
    m}}(\alpha_\ell^\ell)}\right)^{u_1} \Phi_{A_\ell\emb A_{\ell
    m}}(\alpha_\ell).
\end{align*}

We then define
\[
\begin{array}{cccc}
  C_{\ell m/\ell,\, \bm{\ell m}}: & A_{\bm{\ell m}} & \to & A_{\bm{\ell m}} \\
  & x & \mapsto & \left( \cfrac{\Phi_{A_{\ell}\emb A_{\bm{\ell
  m}}}(\alpha_{\ell}^{\ell})}{\Phi_{A_{m}\emb A_{\bm{\ell
  m}}}(\alpha_{m}^{m})}\right)^{u_1} x^{m},
\end{array}
\]
or simply $C_{\ell m/\ell}$ if there is no ambiguity about the domain of the
map. With this definition, we obtain
\[
  C_{\ell m/\ell}(\alpha_{\ell m}) = \Phi_{A_{\ell}\emb A_{\ell
  m}}(\alpha_{\ell}),
\]
and we similarly define $C_{\ell m/m}$, $C_{\ell^2m^2/\ell^2}$, and
$C_{\ell^2m^2/m^2}$. We also define 
\[
  C_{\ell^2m^2/\ell m}(\cdot) = \left(N_{\ell^2/\ell,\,\bm{\ell^2m^2}}\circ
  C_{\ell^2m^2/\ell^2}(\cdot)\right)^{u_2}\times \left(N_{m^2/m,\,\bm{\ell^2m^2}}\circ
  C_{\ell^2m^2/m^2}(\cdot)\right)^{u_1}
\]
and 
\[
  \Phi_{A_{\ell m}\emb A_{\ell^2 m^2}}(\alpha_{\ell m})\eqdef C_{\ell^2m^2/\ell
  m}(\alpha_{\ell^2m^2}).
\]
We then have
\begin{align*}
  N_{\ell^2/\ell,\,\bm{\ell^2m^2}}\circ C_{\ell^2m^2/\ell^2}(\alpha_{\ell^2m^2})
  &= N_{\ell^2/\ell,\,\bm{\ell^2m^2}}(\Phi_{A_{\ell^2}\emb
  A_{\ell^2m^2}}(\alpha_{\ell^2})) \\
  &= \Phi_{A_{\ell^2}\emb
  A_{\ell^2m^2}}(N_{\ell^2/\ell,\,\bm{\ell^2}}(\alpha_{\ell^2})) \\
  &= \Phi_{A_{\ell^2}\emb A_{\ell^2m^2}}\circ\Phi_{A_\ell\emb
    A_{\ell^2}}(\alpha_\ell),
\end{align*}
where the first and the third equality are obtained by definition of
$C_{\ell^2m^2/\ell^2}$ and $N_{\ell^2/\ell}$, and the second equality holds
because the ``norms'' $N$ and the morphims of algebra $\Phi$ commute. We
similarly obtain
\[
  N_{m^2/m,\,\bm{\ell^2m^2}}\circ C_{\ell^2m^2/m^2}(\alpha_{\ell^2m^2}) =
  \Phi_{A_{m^2}\emb A_{\ell^2m^2}}\circ\Phi_{A_m\emb
    A_{m^2}}(\alpha_m),
\]
and so it follows from the definition of $\Phi_{A_{\ell m}\emb
A_{\ell^2m^2}}(\alpha_{\ell m})$ that
\begin{align*}
  \Phi_{A_{\ell m}\emb A_{\ell^2m^2}}(\alpha_{\ell m}) &= \\
  \left(\Phi_{A_{\ell^2}\emb A_{\ell^2m^2}}\circ\Phi_{A_\ell\emb
    A_{\ell^2}}(\alpha_\ell)\right)^{u_2}
    &\times \left(\Phi_{A_{m^2}\emb A_{\ell^2m^2}}\circ\Phi_{A_m\emb
      A_{m^2}}(\alpha_m)\right)^{u_1}.
\end{align*}
Now, we define
\[
  \Phi_{A_{\ell}\emb A_{\ell^2m^2}}\eqdef\Phi_{A_{\ell^2}\emb
  A_{\ell^2m^2}}\circ\Phi_{A_{\ell}\emb A_{\ell^2}},
\]
and we wonder if we have the equality
\[
  \Phi_{A_{\ell}\emb A_{\ell^2m^2}}(\alpha_\ell)\overset{?}{=}\Phi_{A_{\ell
  m}\emb A_{\ell^2m^2}}\circ\Phi_{A_{\ell}\emb A_{\ell m}}(\alpha_\ell).
\]
Then, if we note
\[
  \gamma = \cfrac{\Phi_{A_{\ell}\emb A_{\bm{\ell
  m}}}(\alpha_{\ell}^{\ell})}{\Phi_{A_{m}\emb A_{\bm{\ell
  m}}}(\alpha_{m}^{m})},
\]
we have
\begin{align*}
  \Phi_{A_{\ell m }\emb A_{\ell^2m^2}}(C_{\ell m/\ell}(\alpha_{\ell m})) &=
  \Phi_{A_{\ell m}\emb A_{\ell^2m^2}}(\gamma^{u_1}\alpha_{\ell m}^m)\\
  &= \Phi_{A_{\ell m}\emb A_{\ell^2 m^2}}(\gamma)^{u_1}\Phi_{A_{\ell m}\emb
  A_{\ell^2 m^2}}(\alpha_{\ell m})^{m}\\
  &= \Phi_{A_{\ell^2}\emb A_{\ell^2m^2}}\circ\Phi_{A_{\ell}\emb
  A_{\ell^2}}(\alpha_\ell).
\end{align*}
To obtain the last equality, we use the fact that $\gamma\in k(\zeta_{\ell m})$,
and that the embeddings between the cyclotomic fields $k(\zeta)$ are well
defined and are compatible. In the end we have the wanted equality, that ensures
compatibility between the embeddings of the finite fields $\mathbb{F}_{p^\ell}$,
$\mathbb{F}_{p^{\ell m}}$, and $\mathbb{F}_{p^{\ell^2m^2}}$. Let us now look at
the general case.

\paragraph{General case.} We take three integers $l, m, n\in\mathbb{N}$ such
that
\[
  l\,|\,m\,|\,n
\]
and we assume that they are all coprime with the characteristic $p$. We write 
\begin{align*}
  l &=\prod_{i\in\mathfrak{I}}q_i^{a_i}\\
  m &=\prod_{j\in\mathfrak{J}}q_j^{b_j}\\
  n &=\prod_{k\in\mathfrak{K}}q_k^{c_k}
\end{align*}
the prime decomposition of $l$, $m$ and $n$, where the $q$'s are primes, the
sets $\mathfrak I\subset\mathfrak J\subset\mathfrak K$ are some indices sets,
and the coefficients $a_i$, $b_j$ and $c_k$ are integers. We also note $u_i$
some coefficients such that
\[
  \sum_{i\in\mathfrak I}u_i L_i = 1
\]
with 
\[
  L_i = \prod_{j\in\mathfrak I,\, j\neq i}q_j^{a_j}.
\]
Similarly, we note
\[
  \sum_{j\in\mathfrak J}v_jM_j = 1
\]
with
\[
  M_j = \prod_{k\in\mathfrak J,\,k\neq j}q_k^{b_k}.
\]
We assume that each $q$-adic tower of extensions of $\mathbb{F}_p$ has a set of
well defined ``norms'' that are compatible, along with roots of unity and
solutions of Hilbert 90, as described in
Section~\ref{sec:ladic}\footnote{The case of $q=2$ is not treated yet, as well
as the first steps in the construction of the tower, where the fact that the
cyclotomic fields may not grow is a problem. Still, we will consider these
norms as black boxes and we will explicitly tell when we use one of their
properties.}. We then define
\[
  \zeta_{l}\eqdef\prod_{i\in\mathfrak
  I}(\zeta_{q_i^{a_i}})^{u_i}\footnote{This is (again) an abuse of notation.}
\]
and
\[
  \alpha_l \eqdef \prod_{i\in\mathfrak I}\Phi_{A_{q_i^{a_i}}\emb
A_l}(\alpha_{q_i^{a_i}})^{u_i}
\]
where the images $\Phi_{A_{q_i^{a_i}}\emb A_{l}}(\alpha_{q_i^{a_i}})$ are
computed using Allombert's algorithm. We similarly define $\zeta_m$, $\alpha_m$,
$\zeta_n$ and $\alpha_n$. With these choices, $\zeta_l$ is a primitive $l$-th
root of unity and $\alpha_l$ is a solution of Hilbert 90 in $A_l$ for the root
$1\otimes\zeta_l$. We also obtain, for any $i\in\mathfrak I$:
\begin{align*}
  \alpha_l^{L_i} &= \prod_{j\in\mathfrak I}\Phi_{A_{q_j^{a_j}}\emb
A_l}(\alpha_{q_j^{a_j}})^{u_jL_i}\\
 &= (\prod_{j\in\mathfrak I,\,j\neq i}\Phi_{A_{q_j^{a_j}}\emb
A_l}(\alpha_{q_j^{a_j}})^{u_jL_i}) \Phi_{A_{q_i^{a_i}}\emb
A_l}(\alpha_{q_i^{a_i}})^{u_iL_i} \\
 &= (\prod_{j\in\mathfrak I,\,j\neq i}\Phi_{A_{q_j^{a_j}}\emb
A_l}(\beta_{q_j^{a_j}})^{u_jL_{i,j}}) \Phi_{A_{q_i^{a_i}}\emb
A_l}(\alpha_{q_i^{a_i}})^{1-\sum_{j\in\mathfrak I,\,j\neq i}u_jL_j} \\
&= (\prod_{j\in\mathfrak I,\,j\neq i}\Bigl(\cfrac{\Phi_{A_{q_j^{a_j}}\emb
A_l}(\beta_{q_j^{a_j}})}{\Phi_{A_{q_i^{a_i}}\emb
A_l}(\beta_{q_i^{a_i}})}\Bigr)^{u_jL_{i, j}})\Phi_{A_{q_i^{a_i}}\emb
A_l}(\alpha_{q_i^{a_i}})
\end{align*}
where $L_{i, j}\eqdef\prod_{k\in\mathfrak I,\, k\neq i, j}q_k^{a_k}$ and
$\beta_{q_j^{a_j}}\eqdef
(\alpha_{q_j^{a_j}})^{q_j^{a_j}}\in1\otimes\mathbb{F}_{p}(\zeta_{q_j^{a_j}})$.
Then, if we note
\[
  \gamma_{i} \eqdef \prod_{j\in\mathfrak I,\,j\neq i}\Bigl(\cfrac{\Phi_{A_{q_j^{a_j}}\emb
A_l}(\beta_{q_j^{a_j}})}{\Phi_{A_{q_i^{a_i}}\emb
A_l}(\beta_{q_i^{a_i}})}\Bigr)^{u_jL_{i,j}},
\]
we also define
\[
  C_{l/q_i^{a_i}}(\alpha_l)\eqdef \gamma_i^{-1}\alpha_l^{L_i}
\]
and we have that
\[
  C_{l/q_i^{a_i}}(\alpha_l)=\Phi_{A_{q_i^{a_i}}\emb
  A_{l}}(\alpha_{q_i^{a_i}}).
\]
We similarly define $C_{m/q_j^{b_j}}(\alpha_m)$ for any $j\in\mathfrak J$ and
$C_{n/q_k^{c_k}}(\alpha_n)$ for any $k\in\mathfrak K$. We are now able to define
\[
  C_{m/l}(\alpha_m) \eqdef\prod_{i\in\mathfrak
  I}N_{q_i^{b_i}/q_i^{a_i},\,\bm{m}}(C_{m/q_i^{b_i}}(\alpha_m))^{u_i}
\]
and we will note $\Phi_{A_l\emb
A_m}=\phi_{\mathbb{F}_{p^l}\emb\mathbb{F}_{p^m}}\otimes\phi_{\mathbb{F}_{p}(\zeta_l)\emb\mathbb{F}_p(\zeta_m)}$ the (well defined) morphism of algebra
such that
\[
  \Phi_{A_l\emb A_m}(\alpha_l) \eqdef C_{m/l}(\alpha_m).
\]
It follows that
\begin{align*}
  \Phi_{A_l\emb A_m}(\alpha_l) &= \prod_{i\in\mathfrak
  I}N_{q_i^{b_i}/q_i^{a_i},\,\bm{m}}(\Phi_{A_{q_i^{b_i}}\emb
A_m}(\alpha_{q_i^{b_i}}))^{u_i}\\
&= \prod_{i\in\mathfrak I}\Phi_{A_{q_i^{b_i}}\emb
A_m}(N_{q_i^{b_i}/q_i^{a_i},\,\bm{q_i^{b_i}}}(\alpha_{q_i^{b_i}}))^{u_i}\\
&= \prod_{i\in\mathfrak I}\Phi_{A_{q_i^{b_i}}\emb
A_m}\circ\Phi_{A_{q_i^{a_i}}\emb A_{q_i^{b_i}}}(\alpha_{q_i^{a_i}})^{u_i}
\end{align*}
since the morphisms of algebra $\Phi$ and the ``norms'' $N$ commute. As done
above, we similarly define $C_{n/m}(\alpha_n)$, $C_{n/l}(\alpha_n)$ and the
morphisms $\Phi_{A_m\emb A_n}$ and $\Phi_{A_l\emb A_n}$. We can now state our
compatibility result.
\begin{prop}
We have the equality
\[
  \Phi_{A_l\emb A_n}(\alpha_l) = \Phi_{A_m\emb A_n}\circ\Phi_{A_l\emb
  A_m}(\alpha_l).
\]
\end{prop}
\begin{proof}
  We have 
  \[
    \Phi_{A_m\emb A_n}(\alpha_m)=\prod_{j\in\mathfrak J}\Phi_{A_{q_j^{c_j}}\emb
  A_n}\circ\Phi_{A_{q_j^{b_j}}\emb A_{q_j^{c_j}}}(\alpha_{q_j^{b_j}})^{v_j}
  \]
  and
  \begin{align*}
 \Phi_{A_m\emb A_n}\circ\Phi_{A_l\emb A_m}(\alpha_l)   &= \Phi_{A_m\emb
 A_n}(C_{m/l}(\alpha_m)) \\
 &= \Phi_{A_m\emb A_n}(\prod_{i\in\mathfrak
 I}(N_{q_{i}^{b_i}/q_i^{a_i}}(C_{m/q_i^{b_i}}(\alpha_m))^{u_i})\\
 &= \prod_{i\in\mathfrak I}N_{q_i^{b_i}/q_i^{a_i}}(\Phi_{A_m\emb
 A_n}(C_{m/q_i^{b_i}}(\alpha_m)))^{u_i},
  \end{align*}
  since $\Phi_{A_m\emb A_n}$ is a morsphim of algebra and commute with
  $N_{q_i^{b_i}/q_i^{a_i}}$. For each $i\in\mathfrak I$, we then have
  \begin{align*}
    \Phi_{A_m\emb A_n}(C_{m/q_i^{b_i}}(\alpha_m)) &= \Phi_{A_m\emb
    A_n}(\lambda_i^{-1}\alpha_m^{M_i})\\
    &= \Phi_{A_m\emb A_n}(\lambda_i^{-1})\Phi_{A_m\emb A_n}(\alpha_m)^{M_i},
  \end{align*}
  with
  \[
    \lambda_i\eqdef \prod_{j\in\mathfrak I,\,j\neq i}\Bigl(\cfrac{\Phi_{A_{q_j^{b_j}}\emb
A_m}(\beta_{q_j^{b_j}})}{\Phi_{A_{q_i^{b_i}}\emb
A_m}(\beta_{q_i^{b_i}})}\Bigr)^{v_jM_{i,j}}.
  \]
And finally
  \begin{align*}
    \Phi_{A_m\emb A_n}(\alpha_m)^{M_i} &= \prod_{j\in\mathfrak J}\Phi_{A_{q_j^{c_j}}\emb
  A_n}\circ\Phi_{A_{q_j^{b_j}}\emb A_{q_j^{c_j}}}(\alpha_{q_j^{b_j}})^{v_jM_i}\\
  &= \prod_{j\in\mathfrak J,\,j\neq i}\Bigl(\cfrac{\Phi_{A_{q_j^{c_j}}\emb
  A_n}\circ\Phi_{A_{q_j^{b_j}}\emb A_{q_j^{c_j}}}(\beta_{q_j^{b_j}})}{\Phi_{A_{q_i^{c_i}}\emb
  A_n}\circ\Phi_{A_{q_i^{b_i}}\emb
  A_{q_i^{c_i}}}(\beta_{q_i^{b_i}})}\Bigr)^{v_jM_{i,j}}\\
  &\times\Phi_{A_{q_i^{c_i}}\emb
  A_n}\circ\Phi_{A_{q_i^{b_i}}\emb A_{q_i^{c_i}}}(\alpha_{q_i^{b_i}})\\
  &= \Phi_{A_m\emb A_n}(\lambda_i)\times\Phi_{A_{q_i^{c_i}}\emb
  A_n}\circ\Phi_{A_{q_i^{b_i}}\emb A_{q_i^{c_i}}}(\alpha_{q_i^{b_i}}),
  \end{align*}
  because the elements $\beta$ are all in cyclotomic fields and the embeddings
  are well defined and commute in these fields. In the end we get
  \begin{align*}
  \Phi_{A_m\emb A_n}\circ\Phi_{A_l\emb A_m}(\alpha_l) &= \prod_{i\in\mathfrak
  I}N_{q_i^{b_i}/q_i^{a_i}}(\Phi_{A_{q_i^{c_i}}\emb
A_n}\circ\Phi_{A_{q_i^{b_i}}\emb A_{q_i^{c_i}}}
  (\alpha_{q_i^{b_i}}))^{u_i}\\
&= \prod_{i\in\mathfrak I}\Phi_{A_{q_i^{c_i}}\emb
A_n}\circ\Phi_{A_{q_i^{b_i}}\emb A_{q_i^{c_i}}}\circ\Phi_{A_{q_i^{a_i}}\emb
A_{q_i^{b_i}}}(\alpha_{q_i^{a_i}})^{u_i}\\
&= \prod_{i\in\mathfrak I}\Phi_{A_{q_i^{c_i}}\emb
A_n}\circ\Phi_{A_{q_i^{a_i}}\emb A_{q_i^{c_i}}}(\alpha_{q_i^{a_i}})^{u_i}\\
&= \Phi_{A_l\emb A_n}(\alpha_l).
  \end{align*}
  The last equality is just the definition of $\Phi_{A_l\emb A_n}(\alpha_l)$ and
  the third equality comes from the fact that we know that the morphisms of algebra
  $\Phi$ commute on each $q_i$-adic tower.
\end{proof}
As a corollary, we obtain that the embeddings obtained via the maps $\Phi$ are
compatible.
\section{Techniques using traces}
We notice that we can in fact solve some of our problems using traces too.
\paragraph{In $\ell$-adic towers.} We keep the same notations as in the previous
sections, especially for the roots $\zeta$, the algebras $A$, the solutions of
Hilbert $90$ $\alpha$, the morphisms of algebra $\Phi$, and so on. We construct the roots
of unity in the same way that we used to, meaning that 
\[
  (1\otimes\zeta_{\ell^{n+1}})^\ell = \Phi_{A_{\ell^n}\emb
  A_{\ell^{n+1}}}(1\otimes\zeta_{\ell^n}).
\]
Let $\ell\notin \left\{ 2, p \right\}$ be a prime number and
$n\in\mathbb{N}$ some integer. Let $\alpha_{\ell^{n}}\in A_{\ell^{n}}$ be a
solution of Hilbert $90$ for the root $1\otimes\zeta_{\ell^{n}}$ and $\beta\in
A_{\ell^{n+1}}$ a solution in $A_{\ell^{n+1}}$ for the root
$1\otimes\zeta_{\ell^{n+1}}$. We see that
$\beta^\ell$ is a solution of Hilbert $90$ for the root
$(1\otimes\zeta_{\ell^{n+1}})^\ell=\Phi_{A_{\ell^n}\emb
A_{\ell^{n+1}}}(1\otimes\zeta_{\ell^n})$. What we want is to be able to define
$\alpha_{\ell^{n+1}}\in A_{\ell^{n+1}}$ such that
\[
  (\alpha_{\ell^{n+1}})^\ell = \Phi_{A_{\ell^n}\emb
  A_{\ell^{n+1}}}(\alpha_{\ell^n}),
\]
sadly we do not have any warranty that $\beta^{\ell}\in\Phi_{A_{\ell^n}\emb
A_{\ell^{n+1}}}(A_{\ell^n})$, we only know that
$\beta^{\ell}\in\phi_{\mathbb{F}_{p^{\ell^n}}\emb
\mathbb{F}_{p^{\ell^{n+1}}}}(\mathbb{F}_{p^{\ell^n}})\otimes
k(\zeta_{\ell^{n+1}})$. If $k(\zeta_{\ell^{n+1}})=k(\zeta_{\ell^n})$, then the
last remark is not an obstruction and as already claimed, we can find a suitable
root $\zeta_{\ell^{n+1}}$ and a solution satisfying $\alpha_{\ell^{n+1}}$.
Otherwise we can use a trace-like function that sends the scalars into the
smaller field $\phi_{k(\zeta_{\ell^{n}})\emb
k(\zeta_{\ell^{n+1}})}(k(\zeta_{\ell^{n}}))$, as we did before with the
norm-like function. Let us define
\[
\begin{array}{cccc}
  T_{\ell^{n+1}/\ell^n,\, \bm{\ell^{n+1}}}: & A_{\bm{\ell^{n+1}}} & \to &
  A_{\bm{\ell^{n+1}}} \\
  & \sum_{i}x_i\otimes y_i & \mapsto & \sum_i
  x_i\otimes\Tr_{k(\zeta_{\ell^{n+1}})/k(\zeta_{\ell^{n}})}(y_i)
\end{array}
\]
where $\Tr_{k(\zeta_{\ell^{n+1}})/k(\zeta_{\ell^{n}})}$ denotes the usual trace
for the extension $k(\zeta_{\ell^{n+1}})/k(\zeta_{\ell^{n}})$. If the domain of the map is
clear within the given context, we will omit the second index and only note
$T_{\ell^{n+1}/\ell^n}$. Using the fact that $\sigma\otimes1$ and
$T_{\ell^{n+1}/\ell^{n}}$ commute and that $T_{\ell^{n+1}/\ell^n}$ is
$1\otimes\phi_{k(\zeta_{\ell^n})\emb
k(\zeta_{\ell^{n+1}})}(k(\zeta_{\ell^{n}}))$ linear, we see that
$T_{\ell^{n+1}/\ell^n}(\beta^\ell)$ is still a solution of Hilbert $90$ for the
root $\Phi_{A_{\ell^n}\emb A_{\ell^{n+1}}}(1\otimes\zeta_{\ell^n})$ and that
$T_{\ell^{n+1}/\ell^n}(\beta^\ell)\in\Phi_{A_{\ell^n}\emb
A_{\ell^{n+1}}}(A_{\ell^n})$. It follows that we can find an element
$\alpha_{\ell^{n+1}}\in A_{\ell^{n+1}}$ such that
\[
  T_{\ell^{n+1}/\ell^n}((\alpha_{\ell^{n+1}})^\ell) = \Phi_{A_{\ell^{n}}\emb
  A_{\ell^{n+1}}}(\alpha_{\ell^{n}}).
\]
\paragraph{In composita.}



\section{Embeddings between tensor products}
\label{sec:tensor-products}
In this section, we briefly investigate the nature of the injective morphisms of algebra
between tensor products of the form 
\[
  A\eqdef\mathbb{F}_{p^{\ell_1}}\otimes
\mathbb{F}_{p^{m_1}}
\]
and
\[
  B\eqdef\mathbb{F}_{p^{\ell_2}}\otimes\mathbb{F}_{p^{m_2}}.
\]
We show that
\begin{prop}
 There exist an injective morphism of algebra $\Phi$ from $A$ to $B$ if and only if
 $\ell_1\,|\,\ell_2$ and $m_1\,|\,m_2$. 
\end{prop}
We denote by $\sigma$ the
Frobenius morphism and by $\Phi$ an injective morphism (\ie embedding) of algebra from $A$ to
$B$.

\section{Existence of a compatible lattice}
\label{sec:existence}
We use the same notations as in the previous sections, in particular the
definitions of the maps $\Phi$ and $N$ still hold.
In this section we will investigate the existence of a compatible lattice. We
first define a tower of algebras: let $(d_n)_{n\in\mathbb{N}}$ the sequence
defined by $d_0=1$, $d_1=2$, $d_2=2^2\cdot 3=12$, and more generally
\[
  d_n = \prod_{j=1}^n p_j^{n+1-j}
\]
for $n\geq1$, with $p_j$ the $j$-th prime number. We also define
$D=\left\{d_n\,|\,n\in\mathbb{N}\right\}$. We have 
\[
  d_0\,|\,d_1\,|\,d_2\,|\,\cdots\,|\,d_{n-1}\,|\,d_n\,|\,d_{n+1}\,|\,\cdots,
\]
in other words if $a\leq b$, $d_a\,|\,d_b$. These divisibility properties allow
us to define the tower of algebras
\[
  A_{p-1}\emb A_{p^{2}-1}\emb\cdots\emb A_{p^{d_{n-1}}-1}\emb
  A_{p^{d_n}-1}\emb\cdots.
\]
In particular
\[
  A_{p-1}\eqdef
\mathbb{F}_{p^{p-1}}\otimes\mathbb{F}_{p}(\zeta_{p-1})=\mathbb{F}_{p^{p-1}}\otimes\mathbb{F}_p\cong\mathbb{F}_{p^{p-1}}
\]
and for $n\geq1$
\[
  A_{p^{d_n}-1}=\mathbb{F}_{p^{p^{d_n}-1}}\otimes\mathbb{F}_p(\zeta_{p^{d_n}-1})=\mathbb{F}_{p^{p^{d_n}-1}}\otimes\mathbb{F}_{p^{d_n}}.
\]
We know that $\mathbb{F}_{p^{d_n}}$ contains a primitive $(p^{d_n}-1)$-th root
of unity: aby primitive element of $\mathbb{F}_{p^{d_n}}$ is in fact such a
root. That is why we write
$\mathbb{F}_{p^{d_n}}=\mathbb{F}_{p}(\zeta_{p^{d_n}-1})$, but we have not
specified which roots we consider.
We construct the roots of unity in the following way: we first choose any $(p-1)$-th primitive root of unity and we continue by recurrence. Assume we have
chosen a primitive $(p^{d_{n-1}}-1)$-th root of unity
$\zeta_{p^{d_{n-1}}-1}\in\mathbb{F}_{p^{d_{n-1}}}$
for a certain $n\geq1$. By surjectivity of the norm, we can find an element
$y\in\mathbb{F}_{p^{d_n}}$ such that
\begin{align*}
  N_{\mathbb{F}_{p^{d_n}}/\mathbb{F}_{p^{d_{n-1}}}}(y)&=y^{\frac{p^{d_n}-1}{p^{d_{n-1}}-1}}\\
  &=\zeta_{p^{d_{n-1}}-1}.
\end{align*}
Since $\zeta_{p^{d_{n-1}}-1}$ is primitive, $y$ is also primitive, in other
words $y$ is a $p^{d_n}-1$ primitive root of unity, and we define
$\zeta_{p^{d_n}-1}\eqdef y$1.

We now construct solutions of the Hilbert $90$ problem in each floor of our
tower. We first choose any solution $\alpha_{p-1}\in A_{p-1}$ of Hilbert $90$
associated with the root $\zeta_{p-1}$, and we continue by recurrence. Assume we
have constructed a solution $\alpha_{p^{d_{n-1}}-1}\in A_{p^{d_{n-1}}-1}$ of
Hilbert $90$ associated with the root $\zeta_{p^{d_{n-1}}-1}$ for a certain
$n\geq1$. Then, like in the previous sections, we can find a nonzero solution
$\beta\in A_{p^{d_n}-1}$ of the Hilbert $90$ with the root $\zeta_{p^{d_n}-1}$,
and we get
\[
  N_{p^{d_n}-1/p^{d_{n-1}}-1}(\beta)=(1\otimes\lambda)\Phi_{A_{p^{d_{n-1}}-1}\emb
  A_{p^{d_n}-1}}(\alpha_{p^{d_{n-1}}-1}),
\]
where $1\otimes\lambda$ is an element in the subfield $\Phi_{A_{p^{d_{n-1}}-1}\emb
A_{p^{d_n}-1}}(1\otimes\mathbb{F}_{p^{d_{n-1}}})$. Since the map
$N_{p^{d_n}-1/p^{d_{n-1}}-1}$ acts on $1\otimes\mathbb{F}_{p^{d_n}}$ as the
usual norm $N_{\mathbb{F}_{p^{d_n}}/\mathbb{F}_{p^{d_{n-1}}}}$, we can again use
the surjectivity of the norm to ``correct'' our element $\beta$ as was done in
the previous sections. In the end we obtain an element $\alpha_{p^{d_n}-1}$ such
that
\[
  N_{p^{d_n}-1/p^{d_{n-1}}-1}(\alpha_{p^{d_n}-1})=\Phi_{A_{p^{d_{n-1}}-1}\emb
  A_{p^{d_n}-1}}(\alpha_{p^{d_{n-1}}-1}).
\]
Since our norm-like functions $N$ are transitive, the compatibility of the
morphisms of algebra $\Phi$, and the compatibility of the associated morsphims
of fields $\phi$ follows. But we have only constructed algebras of the form
$A_{p^d-1}$ with $d\in D$, so our lattice has a lot of holes. We can first plug
the holes of the form $A_{p^{n}-1}$ for any $n\in\mathbb{N}$ (not just $n\in
D$), by defining 
\[
  f_n \eqdef \min\left\{ d\in D\,,\,n\,|\,d \right\}
\]
and\footnote{We should say that
  everything is embedded in a big space for that to make sense.}
\begin{align*}
  \zeta_{p^n-1}&\eqdef  N_{p^{f_n}-1/p^n-1}(\zeta_{p^{f_n}-1})\\
  \alpha_{p^n-1}&\eqdef
  N_{p^{f_n}-1/p^n-1}(\alpha_{p^{f_n}-1}).
\end{align*}
Here again, by transitivity of the norms, we achieve compatibility with the
morphisms of algebra $\Phi$ between algebras of the form $A_{p^n-1}$ for $n\in\mathbb{N}$.

Let us now deal with the general case of two integers $\ell\,|\,m$. We define
$a=\omega_{\mathbb{Z}/\ell\mathbb{Z}}(p)$ (respectively
$b=\omega_{\mathbb{Z}/m\mathbb{Z}}(p)$) the multiplicative order of $p$ in
$\mathbb{Z}/\ell\mathbb{Z}$ (resp. in $\mathbb{Z}/m\mathbb{Z}$). We thus define
\[
  \alpha_\ell\eqdef(\alpha_{p^a-1})^{\frac{p^a-1}{\ell}}\in A_{\ell}=\mathbb{F}_{p^\ell}\otimes \mathbb{F}_{p^a}
\]
and
\[
  \alpha_m\eqdef(\alpha_{p^b-1})^{\frac{p^{b}-1}{m}}\in A_m=\mathbb{F}_{p^m}\otimes \mathbb{F}_{p^b}.
\]
If $a=b$ then we have 
\[
  \alpha_{\ell}=\alpha_m^{m/\ell},
\]
but if $a<b$, we need to take additionnal care of the right part of the tensor,
the ``scalar'' part. In that case we have
\[
  \alpha_\ell=N_{p^b-1/p^a-1}((\alpha_{p^b-1})^{\frac{p^a-1}{\ell}})
\]
and we can also write 
\begin{align*}
  (\alpha_{p^b-1})^{\frac{p^a-1}{\ell}}&=(\alpha_{p^b-1})^{\frac{p^b-1}{\ell\cdot\frac{p^b-1}{p^a-1}}}\\
  &\eqdef \alpha_{\ell v}.
\end{align*}
with $v\eqdef\frac{p^b-1}{p^a-1}$.

We see that we have an equality linking $\alpha_\ell$ and
$\alpha_{\ell v}$, though the natural thing would be an
equality linking $\alpha_\ell$ and $\alpha_m$. Unfortunately the map
$N_{p^b-1/p^a-1}=N_{m/\ell}$ send $\zeta_{m}$ to
$\zeta_{m}^{v}$, and this is not necessarily an $\ell$-th primitive root of unity.
By starting elsewhere in the tower, namely $\alpha_{\ell v}$ instead of
$\alpha_{\ell \frac{m}{\ell}}$, we ensure that we arrive on the right root of
unity $\zeta_{\ell}$. Note that we do not necessarily have
$\frac{m}{\ell}\,|\,v$.

In fact, we do not have to go \emph{that high} in the tower, but we will risk to
obtain an other primitive $\ell$-th root of unity and we will have to apply
corrections. Let us write
\[
  \ell=\prod_{j\in\mathcal P_\ell}q_j^{\alpha_j}
\]
the prime decomposition of $\ell$, where $\mathcal P_\ell$ is a set of indices
and the $q_j$'s are primes. We also write
\[
  m=\prod_{j\in\mathcal P_m}q_j^{\beta_j}
\]
and it follows that $\mathcal P_\ell\subset\mathcal P_m$ and that for all
$j\in\mathcal P_\ell$, $\alpha_j\leq\beta_j$. Now denote by $\nu_j(v)$ the
$q_j$-adic valuation of $v$. Our problem happens when, for $j\in\mathcal
P_\ell$,
\[
  \nu_j(v)\neq \beta_j-\alpha_j=\nu_j(m/\ell)
\]
because it eliminates too much (or too little) of the $q_j$ part of the root $\zeta_\ell$. That
is why we can consider to go only as high as 
\[
  \eta=\prod_{j\in\mathcal
  P_\ell}q_j^{\alpha_j+\nu_j(v)}\times\prod_{j\in\mathcal P_m\setminus \mathcal
  P_\ell}q_j^{\beta_j}.
\]
Indeed, with that choice we elimate just as much of each $q_j$ as we want.

\section{TODO}
\begin{itemize}
  \item How to use traces
  \item Using norms with composita
    \begin{itemize}
      \item what can go bad with roots of unity
      \item how to fix it
    \end{itemize}
\end{itemize}
\end{document}
