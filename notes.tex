\documentclass[a4paper,11pt]{article}

\usepackage[utf8]{inputenc}
\usepackage[T1]{fontenc}
\usepackage[english]{babel}
\usepackage{graphicx}
\usepackage{amsmath,amssymb,amsthm,amsopn}
\usepackage{mathrsfs}
\usepackage{graphicx}
\usepackage{array}
\usepackage{makecell}


\usepackage{hyperref}
\hypersetup{
    colorlinks=true,
    linkcolor=blue,
    citecolor=red,
}

%\usepackage[top=1cm,bottom=1cm]{geometry}
%\usepackage{listings}
%\usepackage{xcolor}

\usepackage{tikz}

% Tikz style

\tikzset{round/.style={circle, draw=black, very thick, scale = 0.7}}
\tikzset{arrow/.style={->, >=latex}}
\tikzset{dashed-arrow/.style={->, >=latex, dashed}}

\newtheoremstyle{break}%
{}{}%
{\itshape}{}%
{\bfseries}{}%  % Note that final punctuation is omitted.
{\newline}{}

\newtheoremstyle{sc}%
{}{}%
{}{}%
{\scshape}{}%  % Note that final punctuation is omitted.
{\newline}{}

\theoremstyle{break}
\newtheorem{thm}{Theorem}[section]
\newtheorem{lm}[thm]{Lemma}
\newtheorem{prop}[thm]{Proposition}
\newtheorem{cor}[thm]{Corollary}

\theoremstyle{sc}
\newtheorem{exo}{Exercice}

\theoremstyle{definition}
\newtheorem{defi}[thm]{Definition}
\newtheorem{ex}[thm]{Example}

\theoremstyle{remark}
\newtheorem{rem}[thm]{Remark}

% Math Operators

\DeclareMathOperator{\Card}{Card}
\DeclareMathOperator{\Gal}{Gal}
\DeclareMathOperator{\Id}{Id}
\DeclareMathOperator{\Img}{Im}
\DeclareMathOperator{\Ker}{Ker}
\DeclareMathOperator{\Minpoly}{Minpoly}
\DeclareMathOperator{\Mod}{mod}
\DeclareMathOperator{\Ord}{Ord}
\DeclareMathOperator{\ppcm}{ppcm}
\DeclareMathOperator{\Tr}{Tr}
\DeclareMathOperator{\Vect}{Vect}

% Shortcuts

\newcommand{\dE}{\partial(E)}
\newcommand{\dF}{\partial(F)}
\newcommand{\dG}{\partial(G)}
\newcommand{\diff}{\mathop{}\!\mathrm{d}}
\newcommand{\eg}{\emph{e.g. }}
\newcommand{\emb}{\hookrightarrow}
\newcommand{\embed}[2]{\phi_{#1\hookrightarrow#2}}
\newcommand{\ent}[2]{[\![#1,#2]\!]}
\newcommand{\ie}{\emph{i.e. }}
\newcommand{\ps}[2]{\left\langle#1,#2\right\rangle}





% opening
\title{Some notes on the compatibility of embeddings using Allombert's algorithm}
\author{}



\begin{document}

\maketitle

%\begin{abstract}

%\end{abstract}

%\tableofcontents

%\clearpage

\section{Towers of $\ell$-adic extensions, $\ell\neq2$}
\label{sec:ladic}

Let $p$ be a prime number, we are interested in the tower of $\ell$-adic extensions of
$k=\mathbb{F}_p$, where $\ell\notin\left\{ 2, p
\right\}$ is an other prime
number. For each $i$, we denote by
$\mathbb{F}_{p^{\ell^i}}$ the extension of degree $\ell^i$, $\zeta_{\ell^i}$ a
$\ell^i$-th primitive root of unity and $r_{\ell^i}$ the integer such that
$k(\zeta_{\ell^i})\cong\mathbb{F}_{p^{r_{\ell^i}}}$. We also denote by
$A_{\ell^i}=\mathbb{F}_{p^{\ell^i}}\otimes
k(\zeta_{\ell^i})$ the algebra used to solve Hilbert 90 problems, and
$\alpha_{\ell^i}\in A_{\ell^i}$ the
solution that we use.

We have $r_{\ell}=\omega_{(\mathbb{Z}/\ell\mathbb{Z})^\times}(p)$, where
$\omega_{(\mathbb{Z}/\ell\mathbb{Z})^\times}(p)$ denotes the order of $p$ in
$(\mathbb{Z}/\ell\mathbb{Z})^\times$. If we denote by $\upsilon=v_\ell(p^{r_\ell}-1)$
the $\ell$-adic valuation of $p^{r_\ell}-1$, we get 
\[
  r_\ell = r_{\ell^2} = \cdots = r_{\ell^{\upsilon}},
\]
and if $i\geq\upsilon$, we have $r_{\ell^{i+1}}=\ell r_{\ell^i}$.

Assume we have an integer $n\in\mathbb{N}$, a root
$\zeta_{\ell^n}$ and a solution of Hilbert 90 $\alpha_{\ell^n}\in A_{\ell^n}$. We are interested in
constructing the embedding $\mathbb{F}_{p^{\ell^n}}\emb
\mathbb{F}_{p^{\ell^{n+1}}}$, using Allombert's algorithm. We
first compute 
\[
  g = \gcd(\pi_{\ell^n}(x^\ell), \Phi_{\ell^{n+1}}(x)),
\]
where $\pi_{\ell^n}$ is the minimal polynomial of $\zeta_{\ell^n}$ over $k$ and
$\Phi_{\ell^{n+1}}$ is the $\ell^{n+1}$-th cyclotomic polynomial. We set
$\pi_{\ell^{n+1}}$ to be an irreducicle factor of $g$ and we take
$\zeta_{\ell^{n+1}}$ a root
of $\pi_{\ell^{n+1}}$. We then define
$A_{\ell^{n+1}}=\mathbb{F}_{p^{\ell^{n+1}}}\otimes k(\zeta_{\ell^{n+1}})$ and
\[
\begin{array}{cccc}
  N_{\ell^{n+1}/\ell^n,\, \bm{\ell^{n+1}}}: & A_{\bm{\ell^{n+1}}} & \to &
  A_{\bm{\ell^{n+1}}} \\
  & \gamma & \mapsto & \prod_{j=0}^{\ell-1} (1 \otimes
  \sigma^{jr_{\ell^n}})(\gamma)
\end{array}
\]
where $\sigma$ is the Frobenius automorphism of $k$. If the domain of the map is
clear within the given context, we will omit the second index and only note
$N_{\ell^{n+1}/\ell^n}$. We note that, in particular, given
$\gamma=x\otimes y\in A_{\ell^{n+1}}$ a simple tensor, if
$r_{\ell^{n+1}}=r_{\ell^n}$, we have $N_{\ell^{n+1}/\ell^n}(\gamma) =
\gamma^\ell$, and if $r_{n+1}=\ell r_n$, we have $N_{\ell^{n+1}/\ell^n}(\gamma) = x^\ell\otimes
N_{k(\zeta_{\ell^{n+1}})/k(\zeta_{\ell^n})}(y)$, where
$N_{k(\zeta_{\ell^{n+1}})/k(\zeta_{\ell^n})}$ is the
norm of the extension $k(\zeta_{\ell^{n+1}})$ over $k(\zeta_{\ell^n})$. More generally, if $\gamma\in
A_{\ell^{n+1}}$ is not a simple tensor, we have that
\[
  (1\otimes\sigma^{r_{\ell^n}})(N_{\ell^{n+1}/\ell^n}(\gamma)) =
  N_{\ell^{n+1}/\ell^n}(\gamma),
\]
so we still have 
\[
  N_{\ell^{n+1}/\ell^n}(\gamma)\in \mathbb{F}_{p^{\ell^{n+1}}}\otimes \phi_{k(\zeta_{\ell^n})\emb
k(\zeta_{\ell^{n+1}})}(k(\zeta_n)).
\]

We indeed embed $k(\zeta_{\ell^n})$ in $k(\zeta_{\ell^{n+1}})$ by sending
$\zeta_{\ell^n}$ to
$\zeta_{\ell^{n+1}}^\ell$ and we denote by $\phi_{k(\zeta_{\ell^n})\emb
k(\zeta_{\ell^{n+1}})}$ this embedding. We also note
\[
  \Phi_{A_{\ell^n}\emb
  A_{\ell^{n+1}}}=\phi_{\mathbb{F}_{p^{\ell^n}}\emb\mathbb{F}_{p^{\ell^{n+1}}}}\otimes\phi_{k(\zeta_{\ell^n})\emb
k(\zeta_{\ell^{n+1}})}
\]
an embedding from the algebra $A_{\ell^n}$ in $A_{\ell^{n+1}}$, where $\phi_{\mathbb{F}_{p^{\ell^n}}\emb\mathbb{F}_{p^{\ell^{n+1}}}}$ is some
embedding of $\mathbb{F}_{p^{\ell^n}}$ in $\mathbb{F}_{p^{\ell^{n+1}}}$.
With that choice we have that
\[
  N_{\ell^{n+1}/\ell^n}(1\otimes\zeta_{\ell^{n+1}})=\Phi_{A_{\ell^n}\emb
  A_{\ell^{n+1}}}(1\otimes\zeta_{\ell^n}).\footnote{If $r_{\ell^{n+1}}=r_{\ell^n}$,
  $N_{\ell^{n+1}/\ell^n}(1\otimes\zeta_{n+1})=1\otimes\zeta_{n+1}^\ell$. If
  $r_{\ell^{n+1}}=\ell r_{\ell^n}$, it means that
  there are no roots of $X^\ell-\zeta_{\ell^n}$ in $k(\zeta_{\ell^n})$. Since there are $\ell$-roots of unity in
  $k(\zeta_{\ell^n})$, the polynomial $X^\ell-\zeta_{\ell^n}$ is either irreducible or splits
  into linear polynomial over $k(\zeta_{\ell^n})[X]$. The polynomial $X^\ell
  -\zeta_{\ell^{n+1}}^\ell=X^\ell-\phi_{\mathbb{F}_{p^{\ell^n}}\emb
  \mathbb{F}_{p^{\ell^{n+1}}}}(
  \zeta_{\ell^n})$ is
  the minimal polynomial of $\zeta_{\ell^{n+1}}$ over $\phi_{\mathbb{F}_{p^{\ell^n}}\emb
  \mathbb{F}_{p^{\ell^{n+1}}}}(k(\zeta_{\ell^n}))$ and the product of its
  roots (\ie the norm of $\zeta_{\ell^{n+1}}$) is (up to the sign) the constant
  coefficient, so
  $N_{\ell^{n+1}/\ell^n}(1\otimes\zeta_{\ell^{n+1}})=\Phi_{A_{\ell^n}\emb
  A_{\ell^{n+1}}}(1\otimes\zeta_{\ell^n})$.}
\]
We also see that $N_{\ell^{n+1}/\ell^n}$ and $\sigma\otimes1$ commute, and for any $\alpha,
\beta\in A_{\ell^{n+1}}$, we have that \[
  N_{\ell^{n+1}/\ell^n}(\alpha\beta) =
  N_{\ell^{n+1}/\ell^n}(\alpha)N_{\ell^{n+1}/\ell^n}(\beta).
\]
So if $\beta\in A_{\ell^{n+1}}$ is a nonzero solution of the Hilbert 90 problem
in $A_{\ell^{n+1}}$ for
$1\otimes\zeta_{\ell^{n+1}}$, \ie
\[
  (\sigma\otimes 1)(\beta) = (1\otimes\zeta_{\ell^{n+1}})\beta,
\]
then $N_{\ell^{n+1}/\ell^n}(\beta)$ is a nonzero solution of Hilbert 90 for
$\Phi_{A_{\ell^n}\emb A_{\ell^{n+1}}}(1\otimes\zeta_{\ell^n})$:
\begin{align*}
  (\sigma\otimes1)(N_{\ell^{n+1}/\ell^n}(\beta)) &= N_{\ell^{n+1}/\ell^n}((\sigma\otimes1)(\beta)) \\
  &= N_{\ell^{n+1}/\ell^n}(1\otimes\zeta_{\ell^{n+1}})N_{\ell^{n+1/\ell^n}}(\beta) \\
  &= \Phi_{A_{\ell^n}\emb A_{\ell^{n+1}}}(1\otimes\zeta_n)N_{\ell^{n+1}/\ell^n}(\beta).
\end{align*}
This also implies that $N_{\ell^{n+1}/\ell^n}(\beta)$ is in fact in
$\Phi_{A_{\ell^n}\emb A_{\ell^{n+1}}}(A_{\ell^n})$, so we have that
\[
  N_{\ell^{n+1}/\ell^n}(\beta) = (1\otimes a)\Phi_{A_{\ell^n}\emb
  A_{\ell^{n+1}}}(\alpha_n),
\]
with $1\otimes a\in \Phi_{A_{\ell^n}\emb A_{\ell^{n+1}}}(1\otimes
k(\zeta_{\ell^n}))$ and $a\neq0$, because the solutions of Hilbert 90 in
$\Phi_{A_{\ell^n}\emb A_{\ell^{n+1}}}(A_{\ell^n})$
form a $\phi_{k(\zeta_{\ell^n})\emb k(\zeta_{\ell^{n+1}})}(k(\zeta_{\ell^n}))$ vector space of dimension $1$. We now want to find an
element $1\otimes b\in1\otimes k(\zeta_{\ell^{n+1}})$ such that
\[
N_{\ell^{n+1}/\ell^n}(1\otimes b)=1\otimes a^{-1}.
\]
We distinguish two
cases: $r_{n+1}=\ell r_n$ and $r_{n+1}=r_n$.
Let us first assume that $r_{\ell^{n+1}}=\ell r_{\ell^n}$, it means that
\[
  \phi_{k(\zeta_{\ell^n})\emb k(\zeta_{\ell^{n+1}})}(k(\zeta_{\ell^n}))\subsetneq
k(\zeta_{\ell^{n+1}})
\]
and that $N_{\ell^{n+1}/\ell^{n}}$ operates as the norm
$N_{k(\zeta_{n+1})/k(\zeta_n)}$ on the right side of $A_{\ell^{n+1}}$. Since this norm
is surjective we can find the  element $b$ such that $N_{\ell^{n+1}/\ell^n}(1\otimes b)=1\otimes a^{-1}$ and
we set 
\[
  \alpha_{\ell^{n+1}}\eqdef (1\otimes b)\beta,
\]
in order to have
\[
  N_{\ell^{n+1}/\ell^n}(\alpha_{\ell^{n+1}})=\Phi_{A_{\ell}\emb
  A_{\ell^{n+1}}}(\alpha_{\ell^n}).
\]
In order to compute $b$ we remark that either the element
$a^{-1}$ is a $\ell$-th power in $\phi_{k(\zeta_{\ell^m})\emb
k(\zeta_{\ell^{n+1}})}(k(\zeta_{\ell^n}))$, or the polynomial
$X^\ell-a^{-1}$ is irreducible in $\phi_{k(\zeta_{\ell^m})\emb
k(\zeta_{\ell^{n+1}})}(k(\zeta_{\ell^n}))[X]$, so computing an $\ell$-th
root of $a^{-1}$ is sufficient.

In the case where $r_{\ell^{n+1}}=r_{\ell^n}$, there is only one of the $\ell$ possible roots
$\zeta_{\ell^{n+1}}$ such that $a$ is a $\ell$-th power, at least for $\ell\neq2$.

\section{Composita using norms, $\ell\neq2$}
\label{sec:comp-norm}

The considerations in Section~\ref{sec:ladic} generalize to the case of
embeddings between some composita, but not all of them. Let $m, n\in\mathbb{N}$
two integers such that $m\,|\,n$ and such that for all prime $\ell$ such that
$\ell\,|\,n$, we also have $\ell\,|\,m$. In other words, we want $m$ and $n$ to
have the same set of primes in their factorization. Our goal is to extend the
results of Section~\ref{sec:ladic} to the embedding
$\mathbb{F}_{p^m}\emb\mathbb{F}_{p^n}$. Let us note 
\[
  m = \prod_i \ell_i^{\alpha_i}
\]
the factorization of $m$ into primes. For now, we also assume that for all $i$,
we have 
\[
  \alpha_i\geq h(\ell_i)
\]
where, for any prime $\ell$, 
\[
  h(\ell) = v_\ell(p^{r_\ell}-1)
\]
with $v_\ell$ being the $\ell$-adic valuation and
$r_\ell=\omega_{(\mathbb{Z}/\ell\mathbb{Z})^\times}(p)$ is the multiplicative
order of $p$ in the group $(\mathbb{Z}/\ell\mathbb{Z})^\times$, meaning that
$r_\ell$ is the number such that we have 
\[
  k(\zeta_\ell)=\mathbb{F}_p(\zeta_\ell)=\mathbb{F}_{p^{r_\ell}},
\]
where $\zeta_\ell$ refers to a $\ell$-th primitive root of unity.
In the end, these numbers mean that given an integer $\nu\in\mathbb{N}$, if
$1\leq\nu\leq h(\ell)$, we have
\[
  k(\zeta_{\ell^\nu}) = \mathbb{F}_{p^{r_\ell}},
\]
and if $\nu\geq h(\ell)$, we have
\[
  k(\zeta_{\ell^\nu}) = \mathbb{F}_{p^{\ell^{\nu-h(\ell)}\times r_\ell}}.
\]
Since $r_\ell\leq\ell-1$, these conditions on $n$ and $m$ are equivalent to the fact that
\[
  [k(\zeta_n):k(\zeta_m)] = n/m.
\]

We define $A_n=\mathbb{F}_{p^n}\otimes k(\zeta_n)$ and
\[
\begin{array}{cccc}
  N_{n/m}: & A_n & \to & A_n \\
  & \gamma & \mapsto & \prod_{j=0}^{n/m-1} (1\otimes \sigma^{jd})(\gamma)
\end{array}
\]

where $d$ is the degree of the extension $k(\zeta_m)$ over $k$. We embed
$k(\zeta_m)$ in $k(\zeta_n)$ by sending $\zeta_m$ to $\zeta_n^{n/m}$. With that
choice we have that
\[
  N_{n/m}(1\otimes\zeta_n) = \Phi_{A_m\emb A_n}(1\otimes\zeta_m),
\]
with 
\[
  \Phi_{A_m\emb
  A_n}=\phi_{\mathbb{F}_{p^m}\emb\mathbb{F}_{p^n}}\otimes\phi_{k(\zeta_m)\emb
  k(\zeta_n)}.
\]
Indeed, if $n = \ell m$ for a prime $\ell$ dividing $m$, the arguments of
Section~\ref{sec:ladic} still hold and the more general case follows by
transitivity of the ``norm'' $N$, \ie given $m\,|\,n\,|\,o$, we have
$N_{o/m}=N_{n/m}\circ N_{o/n}$. As in the previous section, we see that
$N_{n/m}$ is multiplicative and that if $\beta\in A_n$ is a nonzero solution of
the Hilbert 90 problem in $A_n$ for $1\otimes\zeta_n$, then $N_{n/m}(\beta)$ is
an element of $A_m$ and is a nonzero solution of Hilbert 90 for
$1\otimes\zeta_m$. Therefore, as the solutions of Hilbert 90 in $A_m$ for
$1\otimes\zeta_m$ form a dimension one $k(\zeta_m)$-vector space and since
$N_{n/m}$ is the usual norm of the extension $k(\zeta_n)$ over $k(\zeta_m)$
(which is surjective) on the right part of $A_n=\mathbb{F}_{p^n}\otimes
k(\zeta_n)$, we can conclude that the compatibility can be achieved, as in
Section~\ref{sec:ladic}.

Now, if we have one $i$ with $\alpha_i\leq h(\ell_i)$, it is not clear how to
extend the results of the last section.

\section{Composita using chinese remainder}
\label{sec:comp-crt}

As we saw in Section~\ref{sec:comp-norm}, if we use only norms, we can only
compatibly embed finite fields $\mathbb{F}_{p^m}\emb\mathbb{F}_{p^n}$ if
$[k(\zeta_n):k(\zeta_m)]=n/m$. We will now present a different technique that
still uses norms on the different towers of $\ell$-adic extensions, but
constructs the composita using chinese remainder. Let us start with a simple
example. We let $\ell\in\mathbb{N}$ and $m\in\mathbb{N}$ be two different prime numbers and we will
investigate the compatibility of the embeddings between $\mathbb{F}_{p^\ell}$,
$\mathbb{F}_{p^{\ell m}}$ and $\mathbb{F}_{p^{\ell^2 m^2}}$. Let $u_1, u_2, v_1,
v_2\in \mathbb{N}$ be integers such that
\[
  u_1\ell+u_2m=1
\]
\[
  v_1\ell^2+v_2m^2=1.
\]
Let $\zeta_\ell$ (resp. $\zeta_m$) be a $\ell$-th (resp. $m$-th) primitive root
of unity. Using the construction described in Section~\ref{sec:ladic}, we denote
by $\zeta_{\ell^2}$ (resp. $\zeta_{m^2}$) an $\ell^2$-th (resp. $m^2$-th)
primitive root of unity and $\alpha_{\ell^2}\in A_{\ell^2}$ (resp.
$\alpha_{m^2}\in A_{m^2}$) a nonzero solution of Hilbert $90$ in $A_{\ell^2}$
for $1\otimes\zeta_{\ell^2}$ (resp. in $A_{m^2}$ for $1\otimes\zeta_{m^2}$) such that
\[
  N_{\ell^2/\ell}(\alpha_{\ell^2})=\Phi_{A_\ell\emb A_{\ell^2}}(\alpha_\ell)
\]
\[
  N_{m^2/m}(\alpha_{m^2})=\Phi_{A_m\emb A_{m^2}}(\alpha_m).
\]
We also define\footnote{This is an abuse of notation.}
\[
  \zeta_{\ell m}=\zeta_\ell^{u_2}\zeta_{m}^{u_1}
\]
and
\[
  \zeta_{\ell^2 m^2}=\zeta_{\ell^2}^{v_2}\zeta_{m^2}^{v_1},
\]
so that $\zeta_{\ell m}^\ell = \zeta_m$, $\zeta_{\ell m}^\ell=\zeta_m$,
$\zeta_{\ell^2m^2}^{\ell^2}=\zeta_{m^2}$ and $\zeta_{\ell^2
m^2}^{m^2}=\zeta_{\ell^2}$, and
\[
  \alpha_{\ell m}=\Phi_{A_\ell\emb A_{\ell m}}(\alpha_\ell)^{u_2}\Phi_{A_m\emb
    A_{\ell m}}(\alpha_m)^{u_1}
\]
\[
  \alpha_{\ell^2 m^2}=\Phi_{A_{\ell^2}\emb A_{\ell^2
  m^2}}(\alpha_{\ell^2})^{v_2}\Phi_{A_{m^2}\emb
  A_{\ell^2 m^2}}(\alpha_{m^2})^{v_1}
\]
where the images of the maps $\Phi$ are obtained using Allombert's algorithm.

With these definitions, we get:
\begin{align*}
  \alpha_{\ell m}^m &= \Phi_{A_\ell\emb A_{\ell m}}(\alpha_\ell)^{u_2m}\Phi_{A_m\emb
    A_{\ell m}}(\alpha_m)^{u_1m} \\
  &= \Phi_{A_\ell\emb A_{\ell m}}(\alpha_\ell)^{1-u_1\ell}\Phi_{A_m\emb
    A_{\ell m}}(\alpha_m)^{u_1m} \\
    &= \left( \cfrac{\Phi_{A_m\emb A_{\ell m}}(\alpha_{m}^m)}{\Phi_{A_\ell\emb A_{\ell
    m}}(\alpha_\ell^\ell)}\right)^{u_1} \Phi_{A_\ell\emb A_{\ell
    m}}(\alpha_\ell).
\end{align*}

We then define
\[
\begin{array}{cccc}
  T_{\ell m/\ell,\, \bm{\ell m}}: & A_{\bm{\ell m}} & \to & A_{\bm{\ell m}} \\
  & x & \mapsto & \left( \cfrac{\Phi_{A_{\ell}\emb A_{\bm{\ell
  m}}}(\alpha_{\ell}^{\ell})}{\Phi_{A_{m}\emb A_{\bm{\ell
  m}}}(\alpha_{m}^{m})}\right)^{u_1} x^{m},
\end{array}
\]
or simply $T_{\ell m/\ell}$ if there is no ambiguity about the domain of the
map. With this definition, we obtain
\[
  T_{\ell m/\ell}(\alpha_{\ell m}) = \Phi_{A_{\ell}\emb A_{\ell
  m}}(\alpha_{\ell}),
\]
and we similarly define $T_{\ell m/m}$, $T_{\ell^2m^2/\ell^2}$, and
$T_{\ell^2m^2/m^2}$. We also define 
\[
  T_{\ell^2m^2/\ell m}(\cdot) = \left(N_{\ell^2/\ell}\circ
  T_{\ell^2m^2/\ell^2}(\cdot)\right)^{u_2}\times \left(\mathbb{N}_{m^2/m}\circ
  T_{\ell^2m^2/m^2}(\cdot)\right)^{u_1}
\]
and 
\[
  \Phi_{A_{\ell m}\emb A_{\ell^2 m^2}}(\alpha_{\ell m})\eqdef T_{\ell^2m^2/\ell
  m}(\alpha_{\ell^2m^2}).
\]


\end{document}
