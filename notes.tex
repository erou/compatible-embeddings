\documentclass[a4paper,11pt]{article}

\usepackage[utf8]{inputenc}
\usepackage[T1]{fontenc}
\usepackage[english]{babel}
\usepackage{graphicx}
\usepackage{amsmath,amssymb,amsthm,amsopn}
\usepackage{mathrsfs}
\usepackage{graphicx}
\usepackage{array}
\usepackage{makecell}


\usepackage{hyperref}
\hypersetup{
    colorlinks=true,
    linkcolor=blue,
    citecolor=red,
}

%\usepackage[top=1cm,bottom=1cm]{geometry}
%\usepackage{listings}
%\usepackage{xcolor}

\usepackage{tikz}

% Tikz style

\tikzset{round/.style={circle, draw=black, very thick, scale = 0.7}}
\tikzset{arrow/.style={->, >=latex}}
\tikzset{dashed-arrow/.style={->, >=latex, dashed}}

\newtheoremstyle{break}%
{}{}%
{\itshape}{}%
{\bfseries}{}%  % Note that final punctuation is omitted.
{\newline}{}

\newtheoremstyle{sc}%
{}{}%
{}{}%
{\scshape}{}%  % Note that final punctuation is omitted.
{\newline}{}

\theoremstyle{break}
\newtheorem{thm}{Theorem}[section]
\newtheorem{lm}[thm]{Lemma}
\newtheorem{prop}[thm]{Proposition}
\newtheorem{cor}[thm]{Corollary}

\theoremstyle{sc}
\newtheorem{exo}{Exercice}

\theoremstyle{definition}
\newtheorem{defi}[thm]{Definition}
\newtheorem{ex}[thm]{Example}

\theoremstyle{remark}
\newtheorem{rem}[thm]{Remark}

% Math Operators

\DeclareMathOperator{\Card}{Card}
\DeclareMathOperator{\Gal}{Gal}
\DeclareMathOperator{\Id}{Id}
\DeclareMathOperator{\Img}{Im}
\DeclareMathOperator{\Ker}{Ker}
\DeclareMathOperator{\Minpoly}{Minpoly}
\DeclareMathOperator{\Mod}{mod}
\DeclareMathOperator{\Ord}{Ord}
\DeclareMathOperator{\ppcm}{ppcm}
\DeclareMathOperator{\Tr}{Tr}
\DeclareMathOperator{\Vect}{Vect}

% Shortcuts

\newcommand{\dE}{\partial(E)}
\newcommand{\dF}{\partial(F)}
\newcommand{\dG}{\partial(G)}
\newcommand{\diff}{\mathop{}\!\mathrm{d}}
\newcommand{\eg}{\emph{e.g. }}
\newcommand{\emb}{\hookrightarrow}
\newcommand{\embed}[2]{\phi_{#1\hookrightarrow#2}}
\newcommand{\ent}[2]{[\![#1,#2]\!]}
\newcommand{\ie}{\emph{i.e. }}
\newcommand{\ps}[2]{\left\langle#1,#2\right\rangle}





% opening
\title{Some notes on the compatibility of embeddings using Allombert's algorithm}
\author{}



\begin{document}

\maketitle

%\begin{abstract}

%\end{abstract}

%\tableofcontents

%\clearpage

\section{Towers of $\ell$-adic extensions, $\ell\neq2$}
\label{sec:ladic}

Let $p$ be a prime number, we are interested in the tower of extensions of
$k=\mathbb{F}_p$ of degree $\ell^i$, where $\ell\notin\left\{ 2, p
\right\}$ is an other prime
number and $i\in\mathbb{N}$ is an integer. For each $i$, we denote by
$K_i=\mathbb{F}_{p^{\ell^i}}$ the extension of degree $\ell^i$, $\zeta_i$ a
$\ell^i$-th primitive root of unity and $r_i$ the integer such that
$k(\zeta_i)\cong\mathbb{F}_{p^{r_i}}$. We also denote by $A_i=K_i\otimes
k(\zeta_i)$ the algebra used to solve Hilbert 90 problems, and $\alpha_i\in A_i$ the
solution that we use.

We have $r_1=\omega_{(\mathbb{Z}/\ell\mathbb{Z})^\times}(p)$, where
$\omega_{(\mathbb{Z}/\ell\mathbb{Z})^\times}(p)$ denotes the order of $p$ in
$(\mathbb{Z}/\ell\mathbb{Z})^\times$. If we denote by $\upsilon=v_\ell(p^{r_1}-1)$
the $\ell$-adic valuation of $p^{r_1}-1$, we get 
\[
  r_1 = r_2 = \cdots = r_{\upsilon},
\]
and if $i\geq\upsilon$, we have $r_{i+1}=\ell r_i$.

Assume we have an integer $n\in\mathbb{N}$, a root
$\zeta_n$ and a solution of Hilbert 90 $\alpha_n\in A_n$. We are interested in
constructing the embedding $K_n\emb K_{n+1}$, using Allombert's algorithm. We
first compute 
\[
  g = \gcd(\pi_n(x^\ell), \Phi_{\ell^{n+1}}(x)),
\]
where $\pi_n$ is the minimal polynomial of $\zeta_n$ over $k$ and
$\Phi_{\ell^{n+1}}$ is the $\ell^{n+1}$-th cyclotomic polynomial. We set
$\pi_{n+1}$ to be an irreducicle factor of $g$ and we take $\zeta_{n+1}$ a root
of $\pi_{n+1}$. We then define $A_{n+1}=K_{n+1}\otimes k(\zeta_{n+1})$ and
\[
\begin{array}{cccc}
  N: & K_{n+1}\otimes k(\zeta_{n+1}) & \to & K_{n+1}\otimes k(\zeta_n) \\
  & x\otimes y & \mapsto & \prod_{j=0}^{\ell-1} x \otimes \sigma^{jr_n}(y)
\end{array}
\]
where $\sigma$ is the Frobenius automorphism of $k$. In other words, given
$\gamma=x\otimes y\in A_{n+1}$, if $r_{n+1}=r_n$, we have $N(\gamma) =
\gamma^\ell$, and if $r_{n+1}=\ell r_n$, we have $N(\gamma) = x^\ell\otimes
N_{k(\zeta_{n+1})/k(\zeta_n)}(y)$, where $N_{k(\zeta_{n+1})/k(\zeta_n)}$ is the
norm of $k(\zeta_{n+1})$ over $k(\zeta_n)$. 

We embed $k(\zeta_n)$ in $k(\zeta_{n+1})$ by sending $\zeta_n$ to
$\zeta_{n+1}^\ell$. With that choice we have that
\[
  N(1\otimes\zeta_{n+1})=1\otimes\zeta_n.\footnote{If $r_{n+1}=r_n$,
  $N(1\otimes\zeta_{n+1})=1\otimes\zeta_{n+1}^\ell$. If $r_{n+1}=\ell r_n$, it means that
there are no roots of $X^\ell-\zeta_n$. Since there are $\ell$-roots of unity in
$k(\zeta_n)$, the polynomial $X^\ell-\zeta_n$ is either irreducible or splits
into linear polynomial over $k(\zeta_n)[X]$. The polynomial $X^\ell -\zeta_n$ is
the minimal polynomial of $\zeta_{n+1}$ over $k(\zeta_n)$ and the product of its
roots (\ie the norm of $\zeta_{n+1}$) is (up to the sign) the constant
coefficient, so $N(1\otimes\zeta_{n+1})=1\otimes\zeta_n$.}
\]
We also see that $N$ and $\sigma\otimes1$ commute, and for any $\alpha,
\beta\in A_{n+1}$, we have that \[
  N(\alpha\beta) = N(\alpha)N(\beta).
\]
So if $\beta\in A_{n+1}$ is a nonzero solution of the Hilbert 90 problem in $A_{n+1}$ for
$1\otimes\zeta_{n+1}$, \ie
\[
  (\sigma\otimes 1)(\beta) = (1\otimes\zeta_{n+1})\beta,
\]
then $N(\beta)$ is a nonzero solution of Hilbert 90 for $1\otimes\zeta_n$:
\begin{align*}
  (\sigma\otimes1)(N(\beta)) &= N((\sigma\otimes1)(\beta)) \\
  &= N(1\otimes\zeta_{n+1})N(\beta) \\
  &= (1\otimes\zeta_n)N(\beta).
\end{align*}
This also implies that $N(\beta)$ is in fact in $A_n$, so we have that
\[
  N(\beta) = (1\otimes a)\alpha_n,
\]
with $1\otimes a\in 1\otimes k(\zeta_n)$ and $a\neq0$, because the solutions of Hilbert 90 in $A_n$
form a $k(\zeta_n)$ vector space of dimension $1$. We now want to find an
element $1\otimes b\in1\otimes k(\zeta_{n+1})$ such that $N(1\otimes b)=1\otimes a^{-1}$. We distinguish two
cases: $r_{n+1}=\ell r_n$ and $r_{n+1}=r_n$.

Let us first assume that $r_{n+1}=\ell r_n$, it means that $k(\zeta_n)\subsetneq
k(\zeta_{n+1})$ and that $N$ operates as the norm
$N_{k(\zeta_{n+1})/k(\zeta_n)}$ on the right side of $A_{n+1}$. Since this norm
is surjective we can find the  element $b$ such that $N(1\otimes b)=1\otimes a^{-1}$ and
we set $\alpha_{n+1}= (1\otimes b)\beta$, in order to have
$N(\alpha_{n+1})=\alpha_n$. In order to compute $b$ we remark that either
$a^{-1}$ is a $\ell$-th power in $k(\zeta_n)$, or the polynomial
$X^\ell-a^{-1}$ is irreducible in $k(\zeta_n)[X]$, so computing an $\ell$-th
root of $a^{-1}$ is sufficient.

In the case where $r_{n+1}=r_n$, there is only one of the $\ell$ possible roots
$\zeta_{n+1}$ such that $a$ is a $\ell$-th power, at least for $\ell\neq2$.

\section{Composita, $\ell\neq2$}

The considerations in Section~\ref{sec:ladic} generalize to the case of
embeddings between some composita, but not all of them. Let $m, n\in\mathbb{N}$
two integers such that $m\,|\,n$ and such that for all prime $\ell$ such that
$\ell\,|\,n$, we also have $\ell\,|\,m$. In other words, we want $m$ and $n$ to
have the same set of primes in their factorization. Our goal is to extend the
results of Section~\ref{sec:ladic} to the embedding
$\mathbb{F}_{p^n}\emb\mathbb{F}_{p^m}$. Let us note 
\[
  n = \prod_i \ell_i^{\alpha_i}
\]
the factorization of $n$ into primes. For now, we also assume that for all $i$,
we have 
\[
  \alpha_i\geq h(\ell_i)
\]
where, for any prime $\ell$, 
\[
  h(\ell) = v_\ell(p^{r_\ell}-1)
\]
with $v_\ell$ being the $\ell$-adic valuation and
$r_\ell=\omega_{(\mathbb{Z}/\ell\mathbb{Z})^\times}(p)$ is the multiplicative
order of $p$ in the group $(\mathbb{Z}/\ell\mathbb{Z})^\times$, meaning that
$r_\ell$ is the number such that we have 
\[
  k(\zeta_\ell)=\mathbb{F}_p(\zeta_\ell)=\mathbb{F}_{p^{r_\ell}},
\]
where $\zeta_\ell$ refers to a $\ell$-th primitive root of unity.
In the end, these numbers mean that given an integer $\nu\in\mathbb{N}$, if
$1\leq\nu\leq h(\ell)$, we have
\[
  k(\zeta_{\ell^\nu}) = \mathbb{F}_{p^{r_\ell}},
\]
and if $\nu\geq h(\ell)$, we have
\[
  k(\zeta_{\ell^\nu}) = \mathbb{F}_{p^{\ell^{\nu-h(\ell)}\times r_\ell}}.
\]
Since $r_\ell\leq\ell-1$, these conditions on $n$ and $m$ are equivalent to the fact that
\[
  [k(\zeta_n):k(\zeta_m)] = n/m.
\]

We define $A_n=\mathbb{F}_{p^n}\otimes k(\zeta_n)$ and
\[
\begin{array}{cccc}
  N_{n/m}: & A_n & \to & \mathbb{F}_{p^n}\otimes k(\zeta_m) \\
  & x\otimes y & \mapsto & \prod_{j=0}^{n/m-1} x \otimes \sigma^{jd}(y)
\end{array}
\]

where $d$ is the degree of the extension $k(\zeta_m)$ over $k$. We embed
$k(\zeta_m)$ in $k(\zeta_n)$ by sending $\zeta_m$ to $\zeta_n^{n/m}$. With that
choice we have that
\[
  N_{n/m}(1\otimes\zeta_n) = 1\otimes\zeta_m.
\]
Indeed, if $n = \ell m$ for a prime $\ell$ dividing $m$, the arguments of
Section~\ref{sec:ladic} still hold and the more general case follows by
transitivity of the ``norm'' $N$, \ie given $m\,|\,n\,|\,o$, we have
$N_{o/m}=N_{n/m}\circ N_{o/n}$. As in the previous section, we see that
$N_{n/m}$ is multiplicative and that if $\beta\in A_n$ is a nonzero solution of
the Hilbert 90 problem in $A_n$ for $1\otimes\zeta_n$, then $N_{n/m}(\beta)$ is
an element of $A_m$ and is a nonzero solution of Hilbert 90 for
$1\otimes\zeta_m$. Therefore, as the solutions of Hilbert 90 in $A_m$ for
$1\otimes\zeta_m$ form a dimension one $k(\zeta_m)$-vector space and since
$N_{n/m}$ is the usual norm of the extension $k(\zeta_n)$ over $k(\zeta_m)$
(which is surjective) on the right part of $A_n=\mathbb{F}_{p^n}\otimes
k(\zeta_n)$, we can conclude that the compatibility can be achieved, as in
Section~\ref{sec:ladic}.

Now, if we have one $i$ with $\alpha_i\leq h(\ell_i)$, it is not clear how to
extend the results of the last section.

\end{document}
