\documentclass[a4paper,11pt]{article}

\usepackage[utf8]{inputenc}
\usepackage[T1]{fontenc}
\usepackage[english]{babel}
\usepackage{graphicx}
\usepackage{amsmath,amssymb,amsthm,amsopn}
\usepackage{mathrsfs}
\usepackage{graphicx}
\usepackage{array}
\usepackage{makecell}


\usepackage{hyperref}
\hypersetup{
    colorlinks=true,
    linkcolor=blue,
    citecolor=red,
}

%\usepackage[top=1cm,bottom=1cm]{geometry}
%\usepackage{listings}
%\usepackage{xcolor}

\usepackage{tikz}

% Tikz style

\tikzset{round/.style={circle, draw=black, very thick, scale = 0.7}}
\tikzset{arrow/.style={->, >=latex}}
\tikzset{dashed-arrow/.style={->, >=latex, dashed}}

\newtheoremstyle{break}%
{}{}%
{\itshape}{}%
{\bfseries}{}%  % Note that final punctuation is omitted.
{\newline}{}

\newtheoremstyle{sc}%
{}{}%
{}{}%
{\scshape}{}%  % Note that final punctuation is omitted.
{\newline}{}

\theoremstyle{break}
\newtheorem{thm}{Theorem}[section]
\newtheorem{lm}[thm]{Lemma}
\newtheorem{prop}[thm]{Proposition}
\newtheorem{cor}[thm]{Corollary}

\theoremstyle{sc}
\newtheorem{exo}{Exercice}

\theoremstyle{definition}
\newtheorem{defi}[thm]{Definition}
\newtheorem{ex}[thm]{Example}

\theoremstyle{remark}
\newtheorem{rem}[thm]{Remark}

% Math Operators

\DeclareMathOperator{\Card}{Card}
\DeclareMathOperator{\Gal}{Gal}
\DeclareMathOperator{\Id}{Id}
\DeclareMathOperator{\Img}{Im}
\DeclareMathOperator{\Ker}{Ker}
\DeclareMathOperator{\Minpoly}{Minpoly}
\DeclareMathOperator{\Mod}{mod}
\DeclareMathOperator{\Ord}{Ord}
\DeclareMathOperator{\ppcm}{ppcm}
\DeclareMathOperator{\Tr}{Tr}
\DeclareMathOperator{\Vect}{Vect}

% Shortcuts

\newcommand{\dE}{\partial(E)}
\newcommand{\dF}{\partial(F)}
\newcommand{\dG}{\partial(G)}
\newcommand{\diff}{\mathop{}\!\mathrm{d}}
\newcommand{\eg}{\emph{e.g. }}
\newcommand{\emb}{\hookrightarrow}
\newcommand{\embed}[2]{\phi_{#1\hookrightarrow#2}}
\newcommand{\ent}[2]{[\![#1,#2]\!]}
\newcommand{\ie}{\emph{i.e. }}
\newcommand{\ps}[2]{\left\langle#1,#2\right\rangle}





% opening
\title{Existence of a lattice of compatibly embedded finite fieds using
Allombert algorithm}
\author{}

\begin{document}

\maketitle

%\begin{abstract}

%\end{abstract}

%\tableofcontents

%\clearpage

\section{Introduction and notations}
\label{sec:intro}

Our goal is to construct a \emph{lattice of compatibly embedded finite fields},
meaning that given a collection of finite fields, we want to be able to compute
embeddings betweem them that are furthermore \emph{compatible}, \ie the diagrams
involving morphisms between finite fields should be commutative. Let $p$ be a
prime number, we denote by $\mathbb{F}_{p^{\ell}}$ the finite field with
$p^\ell$ elements and we also sometimes use $k=\mathbb{F}_p$. Allombert's algorithm usually asks for an extension of
scalars because some roots of unity may be missing, so we note 
\[
  A_n \eqdef \mathbb{F}_{p^n}\otimes\mathbb{F}_{p}(\zeta_n)
\]
where $\zeta_n$ denotes a primitive $n$-th root of unity. We consider all our finite
fields to be embedding in an algebraic closure, so if we have $\ell\,|\,m$ two
integers, we know that
\[
  \mathbb{F}_{p^{\ell}}\emb\mathbb{F}_{p^m}
\]
and we will also consider 
\[
  \mathbb{F}_{p^{\ell}}\subset\mathbb{F}_{p^m}
\]
and
\[
  A_{\ell}\subset A_m.
\]
Throughout the document, we consider solutions of equations of the type
\begin{equation}
  \tag{H90}
 (\sigma\otimes1)(x) = (1\otimes\zeta)x
  \label{h90}
\end{equation}
where $\zeta$ is a primitive root of unity. Properties of the solutions of this
equation are given in Allombert's article. For each algebra $A_n$, we note
$\alpha_n$ a solution of \eqref{h90} for the root $\zeta_n$. We also consider
the norm-like maps
\[
\begin{array}{cccc}
  N_{m/\ell,\, \bm{n}}: & A_{\bm{n}} & \to &
  A_{\bm{n}} \\
  & \gamma & \mapsto & \prod_{j=0}^{\left[ k(\zeta_m)\,:\,k(\zeta_\ell) \right]-1} (1 \otimes
  \sigma^{j\left[ k(\zeta_\ell)\,:\,k \right]})(\gamma)
\end{array}
\]
where $\sigma$ is the Frobenius morphism and the notation $\left[ L\,:\,K
\right]$ denotes the degree of the extension $L/K$. The index $n$ will often be
ignored if it is clear from the context. These maps are multiplicative and
transitive, \ie $N_{m/\ell}\circ N_{n/m} = N_{n/\ell}$.

\section{Existence of a compatible lattice}
\label{sec:existence}
We use the same notations as in the previous sections, in particular the
definitions of the maps $\Phi$ and $N$ still hold.
In this section we will investigate the existence of a compatible lattice. We
first define a tower of algebras: let $(d_n)_{n\in\mathbb{N}}$ the sequence
defined by $d_0=1$, $d_1=2$, $d_2=2^2\cdot 3=12$, and more generally
\[
  d_n = \prod_{j=1}^n p_j^{n+1-j}
\]
for $n\geq1$, with $p_j$ the $j$-th prime number. We also define
$D=\left\{d_n\,|\,n\in\mathbb{N}\right\}$. We have 
\[
  d_0\,|\,d_1\,|\,d_2\,|\,\cdots\,|\,d_{n-1}\,|\,d_n\,|\,d_{n+1}\,|\,\cdots,
\]
in other words if $a\leq b$, $d_a\,|\,d_b$. These divisibility properties allow
us to define the tower of algebras
\[
  A_{p-1}\emb A_{p^{2}-1}\emb\cdots\emb A_{p^{d_{n-1}}-1}\emb
  A_{p^{d_n}-1}\emb\cdots.
\]
In particular
\[
  A_{p-1}\eqdef
\mathbb{F}_{p^{p-1}}\otimes\mathbb{F}_{p}(\zeta_{p-1})=\mathbb{F}_{p^{p-1}}\otimes\mathbb{F}_p\cong\mathbb{F}_{p^{p-1}}
\]
and for $n\geq1$
\[
  A_{p^{d_n}-1}=\mathbb{F}_{p^{p^{d_n}-1}}\otimes\mathbb{F}_p(\zeta_{p^{d_n}-1})=\mathbb{F}_{p^{p^{d_n}-1}}\otimes\mathbb{F}_{p^{d_n}}.
\]
We know that $\mathbb{F}_{p^{d_n}}$ contains a primitive $(p^{d_n}-1)$-th root
of unity: any primitive element of $\mathbb{F}_{p^{d_n}}$ is in fact such a
root. That is why we write
$\mathbb{F}_{p^{d_n}}=\mathbb{F}_{p}(\zeta_{p^{d_n}-1})$, but we have not
specified which roots we consider.
We construct the roots of unity in the following way: we first choose any $(p-1)$-th primitive root of unity and we continue by recurrence. Assume we have
chosen a primitive $(p^{d_{n-1}}-1)$-th root of unity
$\zeta_{p^{d_{n-1}}-1}\in\mathbb{F}_{p^{d_{n-1}}}$
for a certain $n\geq1$. By surjectivity of the norm, we can find an element
$y\in\mathbb{F}_{p^{d_n}}$ such that
\begin{align*}
  N_{\mathbb{F}_{p^{d_n}}/\mathbb{F}_{p^{d_{n-1}}}}(y)&=y^{\frac{p^{d_n}-1}{p^{d_{n-1}}-1}}\\
  &=\zeta_{p^{d_{n-1}}-1}.
\end{align*}
Since $\zeta_{p^{d_{n-1}}-1}$ is primitive, $y$ is also primitive, in other
words $y$ is a $(p^{d_n}-1)$-th primitive root of unity, and we define
$\zeta_{p^{d_n}-1}\eqdef y$.

We now construct solutions of \eqref{h90} in each floor of our
tower. We first choose any solution $\alpha_{p-1}\in A_{p-1}$ of Hilbert $90$
associated with the root $\zeta_{p-1}$, and we continue by recurrence. Assume we
have constructed a solution $\alpha_{p^{d_{n-1}}-1}\in A_{p^{d_{n-1}}-1}$ of
Hilbert $90$ associated with the root $\zeta_{p^{d_{n-1}}-1}$ for a certain
$n\geq1$ and let
$\beta\in A_{p^{d_n}-1}$ be any nonzero solution of \eqref{h90} with the root
$\zeta_{p^{d_n}-1}$.
Then we get
\begin{align*}
  (\sigma\otimes1)(N_{p^{d_n}-1/p^{d_{n-1}}-1}(\beta)) &= N_{p^{d_n}-1/p^{d_{n-1}}-1}((\sigma\otimes1)(\beta)) \\
  &= N_{p^{d_n}-1/p^{d_{n-1}}-1}(1\otimes\zeta_{p^{d_n}-1}) N_{p^{d_n}-1/p^{d_{n-1}}-1}(\beta) \\
  &= (1\otimes\zeta_{p^{d_{n-1}}-1})N_{p^{d_n}-1/p^{d_{n-1}}-1}(\beta),
\end{align*}
meaning that we have
\[
  N_{p^{d_n}-1/p^{d_{n-1}}-1}(\beta)=(1\otimes\lambda)\alpha_{p^{d_{n-1}}-1},
\]
where $1\otimes\lambda$ is an element in the subfield $1\otimes\mathbb{F}_{p^{d_{n-1}}}$. Since the map
$N_{p^{d_n}-1/p^{d_{n-1}}-1}$ acts on $1\otimes\mathbb{F}_{p^{d_n}}$ as the
usual norm $N_{\mathbb{F}_{p^{d_n}}/\mathbb{F}_{p^{d_{n-1}}}}$, we can use
the surjectivity of the norm to ``correct'' our element $\beta$. In the end we
obtain an element $\alpha_{p^{d_n}-1}$ such
that
\[
  N_{p^{d_n}-1/p^{d_{n-1}}-1}(\alpha_{p^{d_n}-1})=
  \alpha_{p^{d_{n-1}}-1}.
\]
Since our norm-like functions $N$ are transitive, we have obtained that 
\[
  N_{p^{b}-1/p^a-1}(\alpha_{p^b-1})=\alpha_{p^a-1}
\]
for any $a,b\in D$ such that $a<b$. But we have only constructed algebras of the form
$A_{p^d-1}$ with $d\in D$, so our lattice has a lot of holes. We can first plug
the holes of the form $A_{p^{n}-1}$ for any $n\in\mathbb{N}$ (not just $n\in
D$), by defining 
\[
  f_n \eqdef \min\left\{ d\in D\,,\,n\,|\,d \right\}
\]
and
\begin{align*}
  \zeta_{p^n-1}&\eqdef  N_{p^{f_n}-1/p^n-1}(\zeta_{p^{f_n}-1})\\
  \alpha_{p^n-1}&\eqdef
  N_{p^{f_n}-1/p^n-1}(\alpha_{p^{f_n}-1}).
\end{align*}
Here again, by transitivity of the norms, we obtain that
\[
  N_{p^{b}-1/p^a-1}(\alpha_{p^b-1})=\alpha_{p^a-1}
\]
for any $a,b\in\mathbb{N}$ such that $a\,|\,b$.



Let us now deal with the general case of two integers $\ell\,|\,m$. We define
$a=\omega_{\mathbb{Z}/\ell\mathbb{Z}}(p)$ (respectively
$b=\omega_{\mathbb{Z}/m\mathbb{Z}}(p)$) the multiplicative order of $p$ in
$\mathbb{Z}/\ell\mathbb{Z}$ (resp. in $\mathbb{Z}/m\mathbb{Z}$). We thus define
\[
  \alpha_\ell\eqdef(\alpha_{p^a-1})^{\frac{p^a-1}{\ell}}\in A_{\ell}=\mathbb{F}_{p^\ell}\otimes \mathbb{F}_{p^a}
\]
and
\[
  \alpha_m\eqdef(\alpha_{p^b-1})^{\frac{p^{b}-1}{m}}\in A_m=\mathbb{F}_{p^m}\otimes \mathbb{F}_{p^b}.
\]
If $a=b$ then we have 
\[
  \alpha_{\ell}=\alpha_m^{m/\ell},
\]
but if $a<b$, we need to take additionnal care of the right part of the tensor,
the ``scalar'' part. In that case we have
\[
  \alpha_\ell=N_{p^b-1/p^a-1}((\alpha_{p^b-1})^{\frac{p^a-1}{\ell}})
\]
and we can also write 
\begin{align*}
  (\alpha_{p^b-1})^{\frac{p^a-1}{\ell}}&=(\alpha_{p^b-1})^{\frac{p^b-1}{\ell\cdot\frac{p^b-1}{p^a-1}}}\\
  &\eqdef \alpha_{\ell v}.
\end{align*}
with $v\eqdef\frac{p^b-1}{p^a-1}$.

We see that we have an equality linking $\alpha_\ell$ and
$\alpha_{\ell v}$, though the natural thing would be an
equality linking $\alpha_\ell$ and $\alpha_m$. Unfortunately the map
$N_{p^b-1/p^a-1}=N_{m/\ell}$ send $\zeta_{m}$ to
$\zeta_{m}^{v}$, and this is not necessarily an $\ell$-th primitive root of unity.
By starting elsewhere in the tower, namely $\alpha_{\ell v}$ instead of
$\alpha_{\ell \frac{m}{\ell}}$, we ensure that we arrive on the right root of
unity $\zeta_{\ell}$. Note that we do not necessarily have
$\frac{m}{\ell}\,|\,v$.

In fact, we do not have to go \emph{that high} in the tower, but we will risk to
obtain an other primitive $\ell$-th root of unity and we will have to apply
corrections. Let us write
\[
  \ell=\prod_{j\in\mathcal P_\ell}q_j^{\alpha_j}
\]
the prime decomposition of $\ell$, where $\mathcal P_\ell$ is a set of indices
and the $q_j$'s are primes. We also write
\[
  m=\prod_{j\in\mathcal P_m}q_j^{\beta_j}
\]
and it follows that $\mathcal P_\ell\subset\mathcal P_m$ and that for all
$j\in\mathcal P_\ell$, $\alpha_j\leq\beta_j$. Now denote by $\nu_j(v)$ the
$q_j$-adic valuation of $v$. Our problem happens when, for $j\in\mathcal
P_\ell$,
\[
  \nu_j(v)\neq \beta_j-\alpha_j=\nu_j(m/\ell)
\]
because it eliminates too much (or too little) of the $q_j$ part of the root $\zeta_\ell$. That
is why we can consider to go only as high as 
\[
  \eta=\prod_{j\in\mathcal
  P_\ell}q_j^{\alpha_j+\nu_j(v)}\times\prod_{j\in\mathcal P_m\setminus \mathcal
  P_\ell}q_j^{\beta_j}.
\]
Indeed, with that choice we elimate just as much of each $q_j$ as we want.

\section{Compatibility conditions: how to climb a lattice?}
\label{sec:climb}

We use the same notations as in Sections \ref{sec:intro} and
\ref{sec:existence}. Assume we have two integers $\ell, m\in\mathbb{N}$,
debote by $d$ their $\gcd$ and $n\in\mathbb{N}$ an other integer such that
$\ell\,|\,n$ and $m\,|\,n$. In this section we imagine that we have a set of solutions of
\eqref{h90} $\alpha_{p^\ell-1}$ and $\alpha_{p^m-1}$ that we have already computed, and we wonder if they have been
well-chosen, \ie if we can find an other solution $\alpha_{p^n-1}$ in a bigger algebra that is
sent via the norm to these solutions. We first need a lemma.

\begin{lm}
  \label{lm:norms}
  Let $x\in (\mathbb{F}_{p^\ell})^\times$ and $y\in(\mathbb{F}_{p^m})^\times$ such that 
  \[
    N_{\mathbb{F}_{p^\ell}/\mathbb{F}_{p^d}}(x)=N_{\mathbb{F}_{p^m}/\mathbb{F}_{p^d}}(y).
  \]
  Then there exists $z\in(\mathbb{F}_{p^n})^\times$ such that
  \[
    N_{\mathbb{F}_{p^n}/\mathbb{F}_{p^\ell}}(z)=x
  \]
  and
  \[
    N_{\mathbb{F}_{p^n}/\mathbb{F}_{p^m}}(z)=y.
  \]
\end{lm}
\begin{proof}
  We denote by $f$ the function:
\[
\begin{array}{cccc}
  f: & (\mathbb{F}_{p^n})^\times & \to &
  (\mathbb{F}_{p^\ell})^\times\times(\mathbb{F}_{p^m})^\times\\
  & z & \mapsto & (\mathbb{N}_{\mathbb{F}_{p^n}/\mathbb{F}_{p^\ell}}(z),
  N_{\mathbb{F}_{p^n}/\mathbb{F}_{p^m}}(z))
\end{array}
\]
and we define the set
\[
  \mathcal E = \left\{ (x, y)\in
    (\mathbb{F}_{p^\ell})^\times\times(\mathbb{F}_{p^m})^\times\,|\,N_{\mathbb{F}_{p^\ell}/\mathbb{F}_{p^d}}(x)=N_{\mathbb{F}_{p^m}/\mathbb{F}_{p^d}}(y)\right\}.
\]
We will prove that $\Img(f)=\mathcal E$, which is exactly a translation of the
lemma.

Assume we have an element $(x,y)\in\Img(f)$, then by transitivity of the norm we
also have $(x,y)\in\mathcal E$, so 
\[
  \Img(f)\subset\mathcal E.
\]
We finish the proof by a cardinality argument. The function $f$ is a group
morphism, so we have 
\[
  \Card\Img(f)=\frac{p^{n}-1}{\Card\Ker(f)}.
\]
Moreover the elements in $\Ker(f)$ are those that are in the kernel of both
norms. The kernel of $\mathbb{N}_{\mathbb{F}_{p^n}/\mathbb{F}_{p^\ell}}$ is a
subgroup of $(\mathbb{F}_{p^n})^\times$ of index $p^{\ell}-1$, and the kernel of
$\mathbb{N}_{\mathbb{F}_{p^n}/\mathbb{F}_{p^m}}$ is a
subgroup of $(\mathbb{F}_{p^n})^\times$ of index $p^{m}-1$. The intersection of
these subgroups is a subgroup of index $\lcm(p^{\ell}-1, p^m-1)$. In the end we
obtain that
\[
  \Card\Img(f) = \lcm(p^\ell-1, p^m-1).
\]
We also have 
\begin{align*}
  \Card\mathcal E &= (p^{d}-1)\frac{p^\ell-1}{p^d-1}\frac{p^m-1}{p^d-1}\\
  &= \frac{(p^\ell-1)(p^m-1)}{p^d-1}\\
  &= \lcm(p^\ell-1, p^m-1)
\end{align*}
because $\gcd(p^\ell-1, p^m-1)=p^d-1$. In the end, we have 
\[
  \Card\Img(f)=\Card\mathcal E,
\]
so
\[
  \Img f = \mathcal E
\]
and this concludes the proof.
\end{proof}

\begin{prop}
  Let $\alpha_{p^\ell-1}\in A_{p^\ell-1}$ and $\alpha_{p^m-1}\in A_{p^m-1}$ be
  two solutions of \eqref{h90}, respectively for the roots $\zeta_{p^\ell-1}$ and
  $\zeta_{p^m-1}$, such that
  \[
    N_{p^\ell-1/p^d-1}(\alpha_{p^\ell-1})=N_{p^m-1/p^d-1}(\alpha_{p^m-1}).
  \]
  Given an algebra $A_{p^n-1}$ with a root $\zeta_{p^n-1}$ such that 
  \[
    N_{p^n-1/p^\ell-1}(1\otimes\zeta_{p^n-1})=1\otimes\zeta_{p^\ell-1}
  \]
  and
  \[
    N_{p^n-1/p^m-1}(1\otimes\zeta_{p^n-1})=1\otimes\zeta_{p^m-1},
  \]
  there exists a solution $\alpha_{p^n-1}$ of \eqref{h90} for the root $\zeta_{p^n-1}$ such that
  \[
    N_{p^n-1/p^\ell-1}(\alpha_{p^n-1})=\alpha_{p^\ell-1}
  \]
  and
  \[
    N_{p^n-1/p^m-1}(\alpha_{p^n-1})=\alpha_{p^m-1}.
  \]
\end{prop}
\begin{proof}
  Let $\beta$ be any nonzero solution of \eqref{h90} for the root
  $\zeta_{p^n-1}$, then we have that
  \[
    N_{p^n-1/p^\ell-1}(\beta)=(1\otimes c_\ell)\alpha_{p^\ell-1}
  \]
  and
  \[
    N_{p^n-1/p^m-1}(\beta)=(1\otimes c_m)\alpha_{p^m-1}.
  \]
  It follows that 
  \[
    N_{F_{p^\ell}/\mathbb{F}_{p^d}}(c_\ell)=N_{\mathbb{F}_{p^m}/\mathbb{F}_{p^d}}(c_m),
  \]
  and by the Lemma~\ref{lm:norms}, we can find an element $c_n$ such that
  \[
    N_{\mathbb{F}_{p^n}/\mathbb{F}_{p^\ell}}(c_n) = c_\ell
  \]
  and
  \[
    N_{\mathbb{F}_{p^n}/\mathbb{F}_{p^m}}(c_n)=c_m.
  \]
 We then define 
 \[
   \alpha_{p^n-1}=(c_n)^{-1}\beta
 \]
 and the proposition follows.

\end{proof}

\end{document}
