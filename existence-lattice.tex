\documentclass[a4paper,11pt]{article}

\usepackage[utf8]{inputenc}
\usepackage[T1]{fontenc}
\usepackage[english]{babel}
\usepackage{graphicx}
\usepackage{amsmath,amssymb,amsthm,amsopn}
\usepackage{mathrsfs}
\usepackage{graphicx}
\usepackage{array}
\usepackage{makecell}


\usepackage{hyperref}
\hypersetup{
    colorlinks=true,
    linkcolor=blue,
    citecolor=red,
}

%\usepackage[top=1cm,bottom=1cm]{geometry}
%\usepackage{listings}
%\usepackage{xcolor}

\usepackage{tikz}

% Tikz style

\tikzset{round/.style={circle, draw=black, very thick, scale = 0.7}}
\tikzset{arrow/.style={->, >=latex}}
\tikzset{dashed-arrow/.style={->, >=latex, dashed}}

\newtheoremstyle{break}%
{}{}%
{\itshape}{}%
{\bfseries}{}%  % Note that final punctuation is omitted.
{\newline}{}

\newtheoremstyle{sc}%
{}{}%
{}{}%
{\scshape}{}%  % Note that final punctuation is omitted.
{\newline}{}

\theoremstyle{break}
\newtheorem{thm}{Theorem}[section]
\newtheorem{lm}[thm]{Lemma}
\newtheorem{prop}[thm]{Proposition}
\newtheorem{cor}[thm]{Corollary}

\theoremstyle{sc}
\newtheorem{exo}{Exercice}

\theoremstyle{definition}
\newtheorem{defi}[thm]{Definition}
\newtheorem{ex}[thm]{Example}

\theoremstyle{remark}
\newtheorem{rem}[thm]{Remark}

% Math Operators

\DeclareMathOperator{\Card}{Card}
\DeclareMathOperator{\Gal}{Gal}
\DeclareMathOperator{\Id}{Id}
\DeclareMathOperator{\Img}{Im}
\DeclareMathOperator{\Ker}{Ker}
\DeclareMathOperator{\Minpoly}{Minpoly}
\DeclareMathOperator{\Mod}{mod}
\DeclareMathOperator{\Ord}{Ord}
\DeclareMathOperator{\ppcm}{ppcm}
\DeclareMathOperator{\Tr}{Tr}
\DeclareMathOperator{\Vect}{Vect}

% Shortcuts

\newcommand{\dE}{\partial(E)}
\newcommand{\dF}{\partial(F)}
\newcommand{\dG}{\partial(G)}
\newcommand{\diff}{\mathop{}\!\mathrm{d}}
\newcommand{\eg}{\emph{e.g. }}
\newcommand{\emb}{\hookrightarrow}
\newcommand{\embed}[2]{\phi_{#1\hookrightarrow#2}}
\newcommand{\ent}[2]{[\![#1,#2]\!]}
\newcommand{\ie}{\emph{i.e. }}
\newcommand{\ps}[2]{\left\langle#1,#2\right\rangle}





% opening
\title{Existence of a lattice of compatibly embedded finite fieds using
Allombert algorithm}
\author{}

\addto{\captionsenglish}{\renewcommand{\abstractname}{Warning}}
\begin{document}

\maketitle

\begin{abstract}
  This document is \emph{not} self-contained. Some facts are probably just
  stated here, and proved in Allombert's original article or in other documents
  written before this one\footnote{See
    \url{https://github.com/erou/compatible-embeddings} to find them.}. Please
    tell me if some of the missing proofs should be here.
\end{abstract}

%\tableofcontents

%\clearpage

\section{Introduction and notations}
\label{sec:intro}

Our goal is to construct a \emph{lattice of compatibly embedded finite fields},
meaning that given a collection of finite fields, we want to be able to compute
embeddings between them that are furthermore \emph{compatible}, \ie the diagrams
involving morphisms between finite fields should be commutative. Let $p$ be a
prime number, we denote by $\mathbb{F}_{p^{\ell}}$ the finite field with
$p^\ell$ elements and we also sometimes use $k=\mathbb{F}_p$. Allombert's algorithm usually asks for an extension of
scalars because some roots of unity may be missing, so we note 
\[
  A_n \eqdef \mathbb{F}_{p^n}\otimes\mathbb{F}_{p}(\zeta_n)
\]
where $\zeta_n$ denotes a primitive $n$-th root of unity. We consider all our finite
fields to be embedded in an algebraic closure, so if we have $\ell\,|\,m$ two
integers, we know that
\[
  \mathbb{F}_{p^{\ell}}\emb\mathbb{F}_{p^m}
\]
and we will also consider 
\[
  \mathbb{F}_{p^{\ell}}\subset\mathbb{F}_{p^m}
\]
and
\[
  A_{\ell}\subset A_m.
\]
Throughout the document, we consider solutions of equations of the type
\begin{equation}
  \tag{H90}
 (\sigma\otimes1)(x) = (1\otimes\zeta)x
  \label{h90}
\end{equation}
where $\zeta$ is a primitive root of unity. Properties of the solutions of this
equation are given in Allombert's article. For each algebra $A_n$, we note
$\alpha_n$ a solution of \eqref{h90} for the root $\zeta_n$. We also consider
the norm-like maps
\[
\begin{array}{cccc}
  \N_{b/a,\, \bm{n}}: & A_{\bm{n}} & \to &
  A_{\bm{n}} \\
  & \gamma & \mapsto & \prod_{j=0}^{b/a-1} (1 \otimes
  \sigma^{ja})(\gamma)
\end{array}
\]
where $\sigma$ is the Frobenius morphism and the notation. The index $n$ will often be
ignored if it is clear from the context. These maps are multiplicative and
transitive, \ie $\N_{b/a}\circ \N_{c/b} = \N_{c/a}$.

\section{Existence of a compatible lattice}
\label{sec:existence}
We use the same notations as in the previous sections, in particular the
definition of the map $\N$ still holds. In this section we will investigate the
existence of a compatible lattice. We first prove the more general result that
follows.
\begin{prop}
  There exists a family of solutions of \eqref{h90}
  $(\alpha_{p^n-1})_{n\in\mathbb{N}}$, corresponding to the roots of unity
  $(\zeta_{p^n-1})_{n\in\mathbb{N}}$, such that, for each $a,b\in\mathbb{N}$
  such that $a\,|\,b$, we have
  \[
    \N_{b/a}(\alpha_{p^b-1})=\alpha_{p^a-1}.
  \]
\end{prop}
\begin{proof}
  We prove the result by induction. First, choose any primitive element
  $\zeta_{p-1}\in\mathbb{F}_p$, and a solution $\alpha_{p-1}\in
  A_{p-1}$ of \eqref{h90} for the root $\zeta_{p-1}$. Then, suppose that we have
  a family of solutions of \eqref{h90} $(\alpha_{p^n-1})_{n\in \left[ d_n
  \right]}$ for the roots
  $(\zeta_{p^n-1})_{n\in \left[ d_n \right]}$, where $\left[ d_n \right]$ is
  the set of all the divisors of some integer $d_n\in\mathbb{N}$, and such that for all $a,
  b\in \left[ d_n \right]$ such that $a\,|\,b$, we have
  \[
    \N_{b/a}(\alpha_{p^b-1})=\alpha_{p^a-1}.
  \]
  Let $d_{n+1}\in\mathbb{N}$ such that $d_n\,|\,d_{n+1}$, consider the
  finite field extension $\mathbb{F}_{p^{d_{n+1}}}/\mathbb{F}_{p^{d_n}}$, and
  let $N:\mathbb{F}_{p^{d_{n+1}}}\to\mathbb{F}_{p^{d_n}}$ the corresponding
  norm. Since $N$ is surjective, we note $y\in\mathbb{F}_{p^{d_{n+1}}}$ such
  that
  \[
    N(y) = \zeta_{p^{d_{n}}}.
  \]
  Because $N(y)=y^{\frac{p^{d_{n+1}}-1}{p^{d_n}-1}}$, we have that the order of
  $y$ is $p^{d_{n+1}}-1$, otherwise the order of $N(y)=\zeta_{p^{d_n}}$ could not
  be $p^{d_n}-1$. Hence we can define a new root $\zeta_{p^{d_{n+1}}-1}=y$, and we
  have that 
  \[
    N(\zeta_{p^{d_{n+1}}-1})=\zeta_{p^{d_n}-1}.
  \]
  Furthermore, the map $\N_{d_{n+1}/d_n}$ acts on
  $1\otimes\mathbb{F}_{p^{d_{n+1}}}$ exactly as $1\otimes N$, so we also have
  \[
    \N_{d_{n+1}/d_n}(1\otimes\zeta_{p^{d_{n+1}}-1}) =
    1\otimes\zeta_{p^{d_{n}}-1}.
  \]
Now let
$\beta\in A_{p^{d_{n+1}}-1}$ be any nonzero solution of \eqref{h90} with the root
$\zeta_{p^{d_{n+1}}-1}$.
Then we get
\begin{align*}
  (\sigma\otimes1)(\N_{d_{n+1}/d_{n}}(\beta)) &= \N_{d_{n+1}/d_{n}}((\sigma\otimes1)(\beta)) \\
  &= \N_{d_{n+1}/d_{n}}(1\otimes\zeta_{p^{d_{n+1}}-1}) \N_{d_{n+1}/d_{n}}(\beta) \\
  &= (1\otimes\zeta_{p^{d_{n}}-1})\N_{d_{n+1}/d_{n}}(\beta),
\end{align*}
meaning that we have
\[
  \N_{d_{n+1}/d_{n}}(\beta)=(1\otimes\lambda)\alpha_{p^{d_{n}}-1},
\]
where $1\otimes\lambda$ is an element in the subfield $1\otimes\mathbb{F}_{p^{d_{n}}}$. Since the map
$\N_{d_{n+1}/d_{n}}$ acts on $1\otimes\mathbb{F}_{p^{d_{n+1}}}$ as the
usual norm $N$, we can use
the surjectivity of the norm to ``correct'' our element $\beta$. In the end we
obtain an element $\alpha_{p^{d_{n+1}}-1}$ such
that
\[
  \N_{d_{n+1}/d_{n}}(\alpha_{p^{d_{n+1}}-1})=
  \alpha_{p^{d_{n}}-1}.
\]
Since our norm-like functions $\N$ are transitive, we have obtained that 
\[
  \N_{b/a}(\alpha_{p^b-1})=\alpha_{p^a-1}
\]
for any $a,b\in \left[ d_n \right]\cup\left\{ d_{n+1} \right\}$ such that
$a\,|\,b$. Additionnaly, for every divisor $d$ of $d_{n+1}$ that is not already in $\left[
d_n \right]$, we define
\[
  \alpha_{p^d-1}\eqdef\N_{d_{n+1}/d}(\alpha_{p^{d_{n+1}}-1}).
\]
In the end, for any $a, b\in\left[ d_{n+1} \right]$ such that
$a\,|\,b$ we have that
\[
  \N_{b/a}(\alpha_{p^b-1})=\alpha_{p^a-1}.
\]
The result follows by considering any sequence $(d_n)_{n\in\mathbb{N}}$ such
that for all $\ell\in\mathbb{N}$, there exists $n_\ell\in\mathbb{N}$ with
$\ell\,|\,d_{n_\ell}$. 
\end{proof}

\section{Existence of a compatible lattice (outdated)}
\label{sec:existence-out}
We use the same notations as in the previous sections, in particular the
definition of the map $\N$ still holds.
In this section we will investigate the existence of a compatible lattice. We
first define a tower of algebras: let $(d_n)_{n\in\mathbb{N}}$ the sequence
defined by $d_0=1$, $d_1=2$, $d_2=2^2\cdot 3=12$, and more generally
\[
  d_n = \prod_{j=1}^n p_j^{n+1-j}
\]
for $n\geq1$, with $p_j$ the $j$-th prime number. We also define
$D=\left\{d_n\,|\,n\in\mathbb{N}\right\}$. We have 
\[
  d_0\,|\,d_1\,|\,d_2\,|\,\cdots\,|\,d_{n-1}\,|\,d_n\,|\,d_{n+1}\,|\,\cdots,
\]
in other words if $a\leq b$, $d_a\,|\,d_b$. These divisibility properties allow
us to define the tower of algebras
\[
  A_{p-1}\emb A_{p^{2}-1}\emb\cdots\emb A_{p^{d_{n-1}}-1}\emb
  A_{p^{d_n}-1}\emb\cdots.
\]
In particular
\[
  A_{p-1}\eqdef
\mathbb{F}_{p^{p-1}}\otimes\mathbb{F}_{p}(\zeta_{p-1})=\mathbb{F}_{p^{p-1}}\otimes\mathbb{F}_p\cong\mathbb{F}_{p^{p-1}}
\]
and for $n\geq1$
\[
  A_{p^{d_n}-1}=\mathbb{F}_{p^{p^{d_n}-1}}\otimes\mathbb{F}_p(\zeta_{p^{d_n}-1})=\mathbb{F}_{p^{p^{d_n}-1}}\otimes\mathbb{F}_{p^{d_n}}.
\]
We know that $\mathbb{F}_{p^{d_n}}$ contains a primitive $(p^{d_n}-1)$-th root
of unity: any primitive element of $\mathbb{F}_{p^{d_n}}$ is in fact such a
root. That is why we write
$\mathbb{F}_{p^{d_n}}=\mathbb{F}_{p}(\zeta_{p^{d_n}-1})$, but we have not
specified which roots we consider.
We construct the roots of unity in the following way: we first choose any $(p-1)$-th primitive root of unity and we continue by recurrence. Assume we have
chosen a primitive $(p^{d_{n-1}}-1)$-th root of unity
$\zeta_{p^{d_{n-1}}-1}\in\mathbb{F}_{p^{d_{n-1}}}$
for a certain $n\geq1$. By surjectivity of the norm, we can find an element
$y\in\mathbb{F}_{p^{d_n}}$ such that
\begin{align*}
  N_{\mathbb{F}_{p^{d_n}}/\mathbb{F}_{p^{d_{n-1}}}}(y)&=y^{\frac{p^{d_n}-1}{p^{d_{n-1}}-1}}\\
  &=\zeta_{p^{d_{n-1}}-1}.
\end{align*}
Since $\zeta_{p^{d_{n-1}}-1}$ is primitive, $y$ is also primitive, in other
words $y$ is a $(p^{d_n}-1)$-th primitive root of unity, and we define
$\zeta_{p^{d_n}-1}\eqdef y$.

We now construct solutions of \eqref{h90} in each floor of our
tower. We first choose any solution $\alpha_{p-1}\in A_{p-1}$ of Hilbert $90$
associated with the root $\zeta_{p-1}$, and we continue by recurrence. Assume we
have constructed a solution $\alpha_{p^{d_{n-1}}-1}\in A_{p^{d_{n-1}}-1}$ of
Hilbert $90$ associated with the root $\zeta_{p^{d_{n-1}}-1}$ for a certain
$n\geq1$ and let
$\beta\in A_{p^{d_n}-1}$ be any nonzero solution of \eqref{h90} with the root
$\zeta_{p^{d_n}-1}$.
Then we get
\begin{align*}
  (\sigma\otimes1)(\N_{d_n/d_{n-1}}(\beta)) &= \N_{d_n/d_{n-1}}((\sigma\otimes1)(\beta)) \\
  &= \N_{d_n/d_{n-1}}(1\otimes\zeta_{p^{d_n}-1}) \N_{d_n/d_{n-1}}(\beta) \\
  &= (1\otimes\zeta_{p^{d_{n-1}}-1})\N_{d_n/d_{n-1}}(\beta),
\end{align*}
meaning that we have
\[
  \N_{d_n/d_{n-1}}(\beta)=(1\otimes\lambda)\alpha_{p^{d_{n-1}}-1},
\]
where $1\otimes\lambda$ is an element in the subfield $1\otimes\mathbb{F}_{p^{d_{n-1}}}$. Since the map
$\N_{d_n/d_{n-1}}$ acts on $1\otimes\mathbb{F}_{p^{d_n}}$ as the
usual norm $N_{\mathbb{F}_{p^{d_n}}/\mathbb{F}_{p^{d_{n-1}}}}$, we can use
the surjectivity of the norm to ``correct'' our element $\beta$. In the end we
obtain an element $\alpha_{p^{d_n}-1}$ such
that
\[
  \N_{d_n/d_{n-1}}(\alpha_{p^{d_n}-1})=
  \alpha_{p^{d_{n-1}}-1}.
\]
Since our norm-like functions $\N$ are transitive, we have obtained that 
\[
  \N_{b/a}(\alpha_{p^b-1})=\alpha_{p^a-1}
\]
for any $a,b\in D$ such that $a<b$. But we have only constructed algebras of the form
$A_{p^d-1}$ with $d\in D$, so our lattice has a lot of holes. We can first plug
the holes of the form $A_{p^{n}-1}$ for any $n\in\mathbb{N}$ (not just $n\in
D$), by defining 
\[
  f_n \eqdef \min\left\{ d\in D\,,\,n\,|\,d \right\}
\]
and
\begin{align*}
  \zeta_{p^n-1}&\eqdef  \N_{f_n/n}(\zeta_{p^{f_n}-1})\\
  \alpha_{p^n-1}&\eqdef
  \N_{f_n/n}(\alpha_{p^{f_n}-1}).
\end{align*}
Here again, by transitivity of the norms, we obtain that
\[
  \N_{b/a}(\alpha_{p^b-1})=\alpha_{p^a-1}
\]
for any $a,b\in\mathbb{N}$ such that $a\,|\,b$.


\section{From $p^n-1$ to any integers $\ell$ and $m$}
Let us now deal with the general case of two integers $\ell\,|\,m$. We define
$a=\omega_{\mathbb{Z}/\ell\mathbb{Z}}(p)$ (respectively
$b=\omega_{\mathbb{Z}/m\mathbb{Z}}(p)$) the multiplicative order of $p$ in
$\mathbb{Z}/\ell\mathbb{Z}$ (resp. in $\mathbb{Z}/m\mathbb{Z}$). We thus define
\[
  \alpha_\ell\eqdef(\alpha_{p^a-1})^{\frac{p^a-1}{\ell}}\in A_{\ell}=\mathbb{F}_{p^\ell}\otimes \mathbb{F}_{p^a}
\]
and
\[
  \alpha_m\eqdef(\alpha_{p^b-1})^{\frac{p^{b}-1}{m}}\in A_m=\mathbb{F}_{p^m}\otimes \mathbb{F}_{p^b}.
\]
If $a=b$ then we have 
\[
  \alpha_{\ell}=(\alpha_m)^{m/\ell},
\]
but if $a<b$, we need to take additionnal care of the right part of the tensor,
the ``scalar'' part. In that case we have
\[
  \alpha_\ell=\N_{b/a}(\alpha_{p^b-1})^{\frac{p^a-1}{\ell}}.
\]

We see that we have an equality linking $\alpha_\ell$ and
$\alpha_{p^b-1}$, though the natural thing would be an
equality linking $\alpha_\ell$ and $\alpha_m$. Unfortunately the map
$\N_{b/a}$ sends $\zeta_{m}$ to
$(\zeta_{m})^{\mu}$, with $\mu\eqdef\frac{p^b-1}{p^a-1}$,
and this is not necessarily an $\ell$-th primitive root of unity.
In fact, we do not have to go \emph{that high} in the tower, but we will risk to
obtain an other primitive $\ell$-th root of unity and we will have to apply
corrections. Let us write
\[
  \ell=\prod_{j\in\mathcal P_\ell}q_j^{u_j}
\]
the prime decomposition of $\ell$, where $\mathcal P_\ell$ is the set of indices
such that the $u_j$ are nonzero and the $q_j$'s are primes. We also write
\[
  m=\prod_{j\in\mathcal P_m}q_j^{v_j}\phantom{and}
  \mu=\prod_{j\in\mathcal P_\mu}q_j^{w_j}
\]
with the sets $\mathcal P_m$ and $\mathcal P_\mu$ defined in the same fashion as
$\mathcal P_\ell$. It follows that $\mathcal P_\ell\subset\mathcal P_m$ and that for all
$j$, $u_j\leq v_j$. The set $\mathcal P_\ell$ is not
necessarily included in $\mathcal P_\mu$. Our problem happens when, for $j\in\mathcal
P_\ell$,
\[
  w_j\neq v_j-u_j
\]
because it eliminates too much (or too little) of the $q_j$ part of the root $\zeta_\ell$. That
is why we can consider to go only as high as 
\[
  \eta=\prod_{j\in\mathcal
  P_\ell}q_j^{u_j+m_j}\times\prod_{j\in\mathcal P_m\setminus \mathcal
  P_\ell}q_j^{v_j},
\]
where $ m_j = \max(w_j, v_j-u_j)$.
Indeed, with that choice we have that $m\,|\,\eta\,|\,p^b-1$ and, noting
\[
  T = \prod_{j\in\mathcal P_m}q_j^{t_j}
\]
with $t_j = m_j-w_j = \max(0, (v_j-u_j)-w_j)$, it follows that
\[
  \forall j\in\mathcal P_\ell,\, \nu_{j}(\frac{\mu\ell T}{\eta}) = w_j + u_j +
  t_j - (u_j + m_j) = 0
\]
where $\nu_j$ is the $q_j$-adic valuation, and
\[
  \forall j\notin \mathcal P_\ell,\,\nu_j(\frac{\mu\ell T}{\eta}) = w_j + 0 +
  t_j - v_j = \max(w_j-v_j, 0).
\]
The integer $\frac{\mu\ell T}{\eta}$ is thus coprime with $\ell$, we denote
by $\iota$ its inverse modulo $\ell$, and we note $e=T\iota$. It follows that
\begin{align*}
  \N_{b/a}(\alpha_\eta^e) &= \N_{b/a}((\alpha_{p^b-1})^{\frac{p^b-1}{\eta}e})\\
  &= (\alpha_{p^a-1})^{\frac{p^b-1}{\eta}e}\\
  &= (\alpha_{p^a-1})^{\frac{(p^a-1)(p^b-1)\ell e}{\ell(p^a-1)\eta}}\\
  &= (\alpha_\ell)^{\frac{\mu\ell e}{\eta}}\\
  &= (\alpha_\ell)^{\kappa\ell+1}\\
  &= (a_\ell)^\kappa\alpha_\ell
\end{align*}
where $a_\ell\eqdef(\alpha_{\ell})^\ell\in\mathbb{F}_{p^a}$ and $\kappa$ is the
quotient of the Euclidean division of $\frac{\mu\ell e}{\eta}$ by $\ell$.

\section{Compatibility conditions: how to climb a lattice?}
\label{sec:climb}

We use the same notations as in Sections \ref{sec:intro} and
\ref{sec:existence}. Assume we have two integers $\ell, m\in\mathbb{N}$,
debote by $d$ their $\gcd$ and $n\in\mathbb{N}$ an other integer such that
$\ell\,|\,n$ and $m\,|\,n$. In this section we imagine that we have a set of solutions of
\eqref{h90} $\alpha_{p^\ell-1}$ and $\alpha_{p^m-1}$ that we have already computed, and we wonder if they have been
well-chosen, \ie if we can find an other solution $\alpha_{p^n-1}$ in a bigger algebra that is
sent via the norm to these solutions. We first need a lemma.

\begin{lm}
  \label{lm:norms}
  Let $x\in (\mathbb{F}_{p^\ell})^\times$ and $y\in(\mathbb{F}_{p^m})^\times$ such that 
  \[
    N_{\mathbb{F}_{p^\ell}/\mathbb{F}_{p^d}}(x)=N_{\mathbb{F}_{p^m}/\mathbb{F}_{p^d}}(y).
  \]
  Then there exists $z\in(\mathbb{F}_{p^n})^\times$ such that
  \[
    N_{\mathbb{F}_{p^n}/\mathbb{F}_{p^\ell}}(z)=x
  \]
  and
  \[
    N_{\mathbb{F}_{p^n}/\mathbb{F}_{p^m}}(z)=y.
  \]
\end{lm}
\begin{proof}
  We denote by $f$ the function:
\[
\begin{array}{cccc}
  f: & (\mathbb{F}_{p^n})^\times & \to &
  (\mathbb{F}_{p^\ell})^\times\times(\mathbb{F}_{p^m})^\times\\
  & z & \mapsto & (\mathbb{N}_{\mathbb{F}_{p^n}/\mathbb{F}_{p^\ell}}(z),
  N_{\mathbb{F}_{p^n}/\mathbb{F}_{p^m}}(z))
\end{array}
\]
and we define the set
\[
  \mathcal E = \left\{ (x, y)\in
    (\mathbb{F}_{p^\ell})^\times\times(\mathbb{F}_{p^m})^\times\,|\,N_{\mathbb{F}_{p^\ell}/\mathbb{F}_{p^d}}(x)=N_{\mathbb{F}_{p^m}/\mathbb{F}_{p^d}}(y)\right\}.
\]
We will prove that $\Img(f)=\mathcal E$, which is exactly a translation of the
lemma.

Assume we have an element $(x,y)\in\Img(f)$, then by transitivity of the norm we
also have $(x,y)\in\mathcal E$, so 
\[
  \Img(f)\subset\mathcal E.
\]
We finish the proof by a cardinality argument. The function $f$ is a group
morphism, so we have 
\[
  \Card\Img(f)=\frac{p^{n}-1}{\Card\Ker(f)}.
\]
Moreover the elements in $\Ker(f)$ are those that are in the kernel of both
norms. The kernel of $\mathbb{N}_{\mathbb{F}_{p^n}/\mathbb{F}_{p^\ell}}$ is a
subgroup of $(\mathbb{F}_{p^n})^\times$ of index $p^{\ell}-1$, and the kernel of
$\mathbb{N}_{\mathbb{F}_{p^n}/\mathbb{F}_{p^m}}$ is a
subgroup of $(\mathbb{F}_{p^n})^\times$ of index $p^{m}-1$. The intersection of
these subgroups is a subgroup of index $\lcm(p^{\ell}-1, p^m-1)$. In the end we
obtain that
\[
  \Card\Img(f) = \lcm(p^\ell-1, p^m-1).
\]
We also have 
\begin{align*}
  \Card\mathcal E &= (p^{d}-1)\frac{p^\ell-1}{p^d-1}\frac{p^m-1}{p^d-1}\\
  &= \frac{(p^\ell-1)(p^m-1)}{p^d-1}\\
  &= \lcm(p^\ell-1, p^m-1)
\end{align*}
because $\gcd(p^\ell-1, p^m-1)=p^d-1$. In the end, we have 
\[
  \Card\Img(f)=\Card\mathcal E,
\]
so
\[
  \Img f = \mathcal E
\]
and this concludes the proof.
\end{proof}

\begin{prop}
  Let $\alpha_{p^\ell-1}\in A_{p^\ell-1}$ and $\alpha_{p^m-1}\in A_{p^m-1}$ be
  two solutions of \eqref{h90}, respectively for the roots $\zeta_{p^\ell-1}$ and
  $\zeta_{p^m-1}$, such that
  \[
    N_{p^\ell-1/p^d-1}(\alpha_{p^\ell-1})=N_{p^m-1/p^d-1}(\alpha_{p^m-1}).
  \]
  Given an algebra $A_{p^n-1}$ with a root $\zeta_{p^n-1}$ such that 
  \[
    N_{p^n-1/p^\ell-1}(1\otimes\zeta_{p^n-1})=1\otimes\zeta_{p^\ell-1}
  \]
  and
  \[
    N_{p^n-1/p^m-1}(1\otimes\zeta_{p^n-1})=1\otimes\zeta_{p^m-1},
  \]
  there exists a solution $\alpha_{p^n-1}$ of \eqref{h90} for the root $\zeta_{p^n-1}$ such that
  \[
    N_{p^n-1/p^\ell-1}(\alpha_{p^n-1})=\alpha_{p^\ell-1}
  \]
  and
  \[
    N_{p^n-1/p^m-1}(\alpha_{p^n-1})=\alpha_{p^m-1}.
  \]
\end{prop}
\begin{proof}
  Let $\beta$ be any nonzero solution of \eqref{h90} for the root
  $\zeta_{p^n-1}$, then we have that
  \[
    N_{p^n-1/p^\ell-1}(\beta)=(1\otimes c_\ell)\alpha_{p^\ell-1}
  \]
  and
  \[
    N_{p^n-1/p^m-1}(\beta)=(1\otimes c_m)\alpha_{p^m-1}.
  \]
  It follows that 
  \[
    N_{F_{p^\ell}/\mathbb{F}_{p^d}}(c_\ell)=N_{\mathbb{F}_{p^m}/\mathbb{F}_{p^d}}(c_m),
  \]
  and by the Lemma~\ref{lm:norms}, we can find an element $c_n$ such that
  \[
    N_{\mathbb{F}_{p^n}/\mathbb{F}_{p^\ell}}(c_n) = c_\ell
  \]
  and
  \[
    N_{\mathbb{F}_{p^n}/\mathbb{F}_{p^m}}(c_n)=c_m.
  \]
 We then define 
 \[
   \alpha_{p^n-1}=(c_n)^{-1}\beta
 \]
 and the proposition follows.

\end{proof}

\section{Compatibility in practice}
\label{sec:practice}

In Section~\ref{sec:existence} we have seen that we can construct a compatible
lattice by choising a solution of~\eqref{h90} in an algebra of high order and
deducting the solutions in all the sub-algebras. The cost of this mode of
operation is too high to use it in practice. Instead, we see how we can
increment our lattice (\ie add embeddings and solutions of~\eqref{h90}) in a compatible way.
Essentially, we have to find a way of telling which solutions of~\eqref{h90} are
coming from a solution in a bigger\footnote{Should we introduce the notion of 
\emph{complete} algebra?} algebra. We need a lemma to help us in this task:
\begin{lm}
  \label{lm:formula}
  Let $n\in\mathbb{N}$ be an integer, $\zeta_{p^n-1}\in\mathbb{F}_{p^n}$ a
  primitive element in $\mathbb{F}_{p^n}$ (\ie a $(p^n-1)$-th primitive root of
  unity), and 
  \[
    A_{p^n-1} = \mathbb{F}_{p^{p^n-1}}\otimes\mathbb{F}_{p^n}.
  \]
  Denote by $\alpha_{p^n-1}\in A_{p^{n}-1}$ a solution of~\eqref{h90} for the
  root $\zeta_{p^n-1}$, then we have that
  \[
    a_{p^n-1} = (\alpha_{p^n-1})^{p^n-1} = (1\otimes\zeta_{p^n-1})^n.
  \]
\end{lm}
\begin{proof}
  Since $\alpha_{p^n-1}$ is a solution of~\eqref{h90} for the root
  $\zeta_{p^n-1}$, we have that
  \[
    (\sigma\otimes 1)(\alpha_{p^n-1}) =
    (1\otimes\zeta_{p^n-1})\alpha_{p^n-1}.
  \]
  It follows that
  \begin{align*}
    (1\otimes\zeta_{p^n-1})^n \alpha_{p^n-1} &= (\sigma^n\otimes1)(\alpha_{p^n-1})\\
    &= (\sigma^n\otimes\sigma^n)(\alpha_{p^n-1})\\
    &= (\alpha_{p^n-1})^{p^n}\\
    &= a_{p^n-1}\cdot\alpha_{p^n-1}.
  \end{align*}
  Therefore, we have the equality
  \[
    ( (1\otimes\zeta_{p^n-1})^n - a_{p^n-1})\alpha_{p^n-1} = 0,
  \]
  and since $1\otimes\zeta_{p^n-1}$ and $a_{p^n-1}$ are both elements of the
  subfield $1\otimes\mathbb{F}_{p^n}$, we conclude that 
  \[
    a_{p^n-1} = (1\otimes\zeta_{p^n-1})^n.
  \]
\end{proof}
Lemma~\ref{lm:formula} gives a necessary condition for solutions of~\eqref{h90}
to be derived from a solution in a bigger algebra. We can see that this condition is in fact
also sufficient. Thus, in the following, for some given integer
$\ell\in\mathbb{N}$ and $A_\ell =
\mathbb{F}_{p^\ell}\otimes\mathbb{F}_{p}(\zeta_\ell) =
\mathbb{F}_{p^\ell}\otimes\mathbb{F}_{p^a}$, we shall consider solutions of the
equation
\begin{equation}
  \tag{H90$^\star$}
  (\sigma\otimes1)(x) = (1\otimes\zeta_\ell)x,\,x^\ell =
  (1\otimes\zeta_{p^a-1})^a
  \label{h90s}
\end{equation}
with $\zeta_\ell=(\zeta_{p^a-1})^{\frac{p^a-1}{\ell}}$.

The solutions of~\eqref{h90s} are defined up to a $\ell$-th root of unity, if 
\[
  \alpha_\ell = \sum_{j=0}^{a-1} x_j\otimes \zeta_\ell^j
\]
is a solution, the other solutions are then given by 
\[
  (1\otimes\zeta_\ell)^u\alpha_\ell = \sum_{j=0}^{a-1} \sigma^u(x_j)\otimes
  \zeta_{\ell}^j
\]
for some integer $u$. Thus we see that taking the first coordinate of a solution
of~\eqref{h90s} for the root $\zeta_\ell$ in the basis $(1\otimes\zeta_\ell^j)_j$ (as done in Allombert's
algorithm) defines the same element (up to isomorphism) in
$\mathbb{F}_{p^\ell}$.

Let us investigate the easiest case of compatibility first. Let
$\ell\,|\,m\,|\,n$ be three integers such that
\[
 a \eqdef \omega_{(\mathbb{Z}/\ell\mathbb{Z})^\times}(p) =
  \omega_{(\mathbb{Z}/m\mathbb{Z})^\times}(p) =
  \omega_{(\mathbb{Z}/n\mathbb{Z})^\times}(p).
\]
In other words, the right part of the tensor algebras $A_\ell$, $A_m$ and $A_n$
is $\mathbb{F}_{p^a}$ in the three cases. We assume that we already have a
primitive element $\zeta_{p^a-1}\in\mathbb{F}_{p^a}$ and we define $\zeta_\ell$
(resp. $\zeta_m$, $\zeta_n$) by 
\[
  \zeta_\ell = (\zeta_{p^a-1})^\frac{p^a-1}{\ell}\text{ (resp. }\zeta_m =
  (\zeta_{p^a-1})^\frac{p^a-1}{m},\,\zeta_n =
  (\zeta_{p^a-1})^\frac{p^a-1}{n}  \text{)}.
\]
We then also define $\alpha_\ell\in A_\ell$ (resp. $\alpha_m\in A_m$,
$\alpha_n\in A_n$) a solution
of~\eqref{h90s} for the root $\zeta_\ell$ (resp. $\zeta_m$, $\zeta_n$).
We finally define 
\[
\begin{array}{cccc}
  f: & \mathbb{F}_{p^\ell} & \to &\mathbb{F}_{p^m} \\
  & \lfloor \alpha_{\ell}\rfloor_\ell & \mapsto & \lfloor
(\alpha_{m})^{m/\ell}\rfloor_\ell
\end{array}
\]
with $\lfloor x \rfloor_\ell$ denoting the first coordinate of $x$ in the basis
$(1\otimes\zeta_{\ell}^j)_j$, and we similarly define
\[
\begin{array}{cccc}
  g: & \mathbb{F}_{p^m} & \to &\mathbb{F}_{p^n} \\
  & \lfloor \alpha_{m}\rfloor_m & \mapsto & \lfloor
(\alpha_{n})^{n/m}\rfloor_m
\end{array}
\]
and
\[
\begin{array}{cccc}
  h: & \mathbb{F}_{p^\ell} & \to &\mathbb{F}_{p^n} \\
  & \lfloor \alpha_{\ell}\rfloor_\ell & \mapsto & \lfloor
(\alpha_{m})^{n/\ell}\rfloor_\ell.
\end{array}
\]
\begin{prop}
  The maps $f$, $g$, and $h$ define embeddings and are compatible, \ie
  \[
    g\circ f = h.
  \]
\end{prop}
\begin{proof}
  Let $\Phi_{A_\ell\emb A_m}' =
\phi_{\mathbb{F}_{p^\ell}\emb\mathbb{F}_{p^m}}'\otimes\embed{\mathbb{F}_{p}(\zeta_\ell)}{\mathbb{F}_{p}(\zeta_m)}$
be a morphism of algebra, with $\phi_{\mathbb{F}_{p^\ell}\emb\mathbb{F}_{p^m}}'$
an unknown morphism and 
$\embed{\mathbb{F}_{p}(\zeta_\ell)}{\mathbb{F}_{p}(\zeta_m)}$ the morphism
sending $\zeta_\ell$ to $(\zeta_m)^{m/\ell}$. We know that
\[
  (\alpha_m)^{m/\ell} = (1\otimes\lambda)\Phi_{A_\ell\emb A_m}'(\alpha_\ell) 
\]
for some $\lambda\in\mathbb{F}_{p^a}$. Moreover we also have 
\[
  a_m = \Phi_{A_\ell\emb A_m}'(a_l)
\]
so $\lambda^\ell = 1$, and finally we get that
\begin{align*}
  (\alpha_m)^{m/\ell} &= \Phi_{A_\ell\emb
  A_m}'((1\otimes\zeta_\ell^u)\alpha_\ell) \\
  &= \Phi_{A_\ell\emb
  A_m}'((\sigma^u\otimes 1)(\alpha_\ell)) \\
  &= \Phi_{A_\ell\emb A_m}(\alpha_\ell)
\end{align*}
for some integer $u$, and with
\begin{align*}
 \Phi_{A_\ell\emb A_m} &=
(\phi_{\mathbb{F}_{p^\ell}\emb\mathbb{F}_{p^m}}'\circ\sigma^u)\otimes\embed{\mathbb{F}_{p}(\zeta_\ell)}{\mathbb{F}_{p}(\zeta_m)}\\
&\eqdef
\phi_{\mathbb{F}_{p^\ell}\emb\mathbb{F}_{p^m}}\otimes\embed{\mathbb{F}_{p}(\zeta_\ell)}{\mathbb{F}_{p}(\zeta_m)}.
\end{align*}
We similarly define 
\[
  \Phi_{A_m\emb
  A_n}\eqdef\phi_{\mathbb{F}_{p^m}\emb\mathbb{F}_{p^n}}\otimes\embed{\mathbb{F}_{p}(\zeta_m)}{\mathbb{F}_{p}(\zeta_n)}
\]
and
\[
  \Phi_{A_\ell\emb
  A_n}\eqdef\phi_{\mathbb{F}_{p^\ell}\emb\mathbb{F}_{p^n}}\otimes\embed{\mathbb{F}_{p}(\zeta_\ell)}{\mathbb{F}_{p}(\zeta_n)}.
\]
We finally note that $f = \embed{\mathbb{F}_{p^\ell}}{\mathbb{F}_{p^m}}$, 
$g = \embed{\mathbb{F}_{p^m}}{\mathbb{F}_{p^n}}$, 
$h = \embed{\mathbb{F}_{p^\ell}}{\mathbb{F}_{p^n}}$ and
\[
(\Phi_{A_m\emb A_n}\circ\Phi_{A_\ell\emb A_m})(\alpha_\ell) = \Phi_{A_\ell\emb
A_n}(\alpha_\ell),
\]
therefore it follows that
\[
  g\circ f = h.
\]
\end{proof}

\end{document}
