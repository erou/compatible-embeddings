\documentclass[a4paper,11pt]{article}

\usepackage[utf8]{inputenc}
\usepackage[T1]{fontenc}
\usepackage[english]{babel}
\usepackage{graphicx}
\usepackage{amsmath,amssymb,amsthm,amsopn}
\usepackage{mathrsfs}
\usepackage{graphicx}
\usepackage{array}
\usepackage{makecell}


\usepackage{hyperref}
\hypersetup{
    colorlinks=true,
    linkcolor=blue,
    citecolor=red,
}

%\usepackage[top=1cm,bottom=1cm]{geometry}
%\usepackage{listings}
%\usepackage{xcolor}

\usepackage{tikz}

% Tikz style

\tikzset{round/.style={circle, draw=black, very thick, scale = 0.7}}
\tikzset{arrow/.style={->, >=latex}}
\tikzset{dashed-arrow/.style={->, >=latex, dashed}}

\input{thmstyle.tex}
% Math Operators

\DeclareMathOperator{\Card}{Card}
\DeclareMathOperator{\Gal}{Gal}
\DeclareMathOperator{\Id}{Id}
\DeclareMathOperator{\Img}{Im}
\DeclareMathOperator{\Ker}{Ker}
\DeclareMathOperator{\Minpoly}{Minpoly}
\DeclareMathOperator{\Mod}{mod}
\DeclareMathOperator{\Ord}{Ord}
\DeclareMathOperator{\ppcm}{ppcm}
\DeclareMathOperator{\Tr}{Tr}
\DeclareMathOperator{\Vect}{Vect}

% Shortcuts

\newcommand{\dE}{\partial(E)}
\newcommand{\dF}{\partial(F)}
\newcommand{\dG}{\partial(G)}
\newcommand{\diff}{\mathop{}\!\mathrm{d}}
\newcommand{\eg}{\emph{e.g. }}
\newcommand{\emb}{\hookrightarrow}
\newcommand{\embed}[2]{\phi_{#1\hookrightarrow#2}}
\newcommand{\ent}[2]{[\![#1,#2]\!]}
\newcommand{\ie}{\emph{i.e. }}
\newcommand{\ps}[2]{\left\langle#1,#2\right\rangle}
\newcommand{\eqdef}{\overset{\text{def}}{=}}


% opening
\title{Notes on compatible composita}
\author{}



\begin{document}

\maketitle

%\begin{abstract}

%\end{abstract}

%\tableofcontents

%\clearpage

In this document, we investigate compatibility questions in the case of the
compositum of two finite fields $\mathbb{F}_{p^m}$ and $\mathbb{F}_{p^n}$, where
the integers $m$ and $n$ are coprime. In that case, the compositum is
$\mathbb{F}_{p^{mn}}$. We denote by $\zeta_m$ (resp. $\zeta_n$) a primitive $m$-th 
root of unity (resp. $n$-th). We set $\alpha_m$ a solution of Hilbert 90 in
$\mathbb{F}_{p^m}\otimes\mathbb{F}_{p}(\zeta_m)$ for the root $1\otimes\zeta_m$,
and we similarly set $\alpha_n$. Let now $u, v$ be integers such that
\[
  un+vm = 1.
\]
We now set
\[
  \zeta=\zeta_m^u\zeta_n^v
\]
and 
\[
  \alpha=\alpha_m^u\alpha_n^v,
\]
such that $\zeta$ is a primitive $mn$-th root of unity, $\zeta^m=\zeta_n$,
$\zeta^n=\zeta_m$ and $\alpha$ is a
solution of Hilbert 90 in $\mathbb{F}_{p^{mn}}\otimes\mathbb{F}_{p}(\zeta)$ for
the root $1\otimes\zeta$. We also have that
\[
  (\sigma\otimes\Id)(\alpha^n) = (1\otimes\zeta_m)\alpha^n
\]
and
\[
  (\sigma\otimes\Id)(\alpha^m) = (1\otimes\zeta_n)\alpha^m
\]
but not necessarily $\alpha^n=\alpha_m$ and $\alpha^m=\alpha_n$. In fact we have 
\begin{align*}
  \alpha^n &= \alpha_m^{un}\alpha_n^{vn} \\
  &= \alpha_m^{1-vm}a_n^v \\
  &= \cfrac{a_n^v}{a_m^v}\alpha_m
\end{align*}
where $a_m=\alpha_m^m\in\mathbb{F}_p(\zeta_m)$ and $a_n =
\alpha_n^n\in\mathbb{F}_p(\zeta_n)$. We also get
\[
  \alpha^m = \cfrac{a_m^u}{a_n^u}\alpha_n
\]
in the same way.
\end{document}
